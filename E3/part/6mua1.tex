\subsection{Uncertainty of the slope of $T^2\ vs.\ M$}
The uncertainty of one period is $u_T/10=1\times10^{-5}s$ and the uncertainty of mass is
\[
\begin{split}
    M&=m_{objI}+\frac{1}{3}m_{spr1}+\frac{1}{3}m_{spr2}+m_i,\\
    u_M&=\sqrt{
     (\frac{\partial M}{\partial m_{objI}})^2(u_{m_{objI}})^2
    +(\frac{\partial M}{\partial m_{spr1}})^2(u_{m_{spr1}})^2
    +(\frac{\partial M}{\partial m_{spr2}})^2(u_{m_{spr2}})^2
    +(\frac{\partial M}{\partial m_i})^2(u_{m_i})^2
    }\\
    &=\sqrt{
     (1)^2(u_{m_{objI}})^2
    +(\frac{1}{3})^2(u_{m_{spr1}})^2
    +(\frac{1}{3})^2(u_{m_{spr2}})^2
    +(1)^2(u_{m_i})^2
    }
\end{split}
\]

As data provided 
$m_{objI}=(176.87\pm0.01)\times10^{-3}kg$, $m_{spr1}=(11.23\pm0.01)\times10^{-3}kg$, $m_{spr2}=(10.54\pm0.01)\times10^{-3}kg$ and $m_1=(4.65\pm0.01)\times10^{-3}kg$

\[
\begin{split}
    u_M & = \sqrt{0.00001^2+0.00001^2/9+0.00001^2/9+0.00001^2}=0.0000153\\
        & = 1.5\times10^{-5}kg.
\end{split}
\]

For $T^2$,
$$   u_{T^2}=\sqrt{(2T)^2(u_T)^2}=2Tu_T$$

From the formula $T=2\pi\sqrt{\frac{M}{k}}$ 
we can drive the slope of $T^2\ vs.\ M$ is

$$	slope=\frac{T^2}{M}=\frac{4\pi^2}{k}$$

where k is the effective spring constant. In this experiment, the effective spring constant can be calculated by

$$  F =k_1\Delta x+k_2\Delta x  =k_{eff}\Delta x   $$

%NOTE: finish here

\[
\begin{split}
u_{slope} & = \sqrt{
	(\frac{\partial slope}{\partial slope_{hor}})^2\cdot(u_{slope_{hor}})^2
	+(\frac{\partial slope}{\partial slope_{inc1}})^2\cdot(u_{sloinc1})^2
	+(\frac{\partial slope}{\partial slope_{inc2}})^2\cdot(u_{slope_{inc2}})^2} \\
& = \sqrt{
	(\frac{1}{3})^2 \cdot (0.22)^2
	+(\frac{1}{3})^2 \cdot (0.05)^2
	+(\frac{1}{3})^2 \cdot (0.22)^2} = 0.18s^2/kg\\
\end{split}
\]

$$ u_{r,slope} = \frac{u_{slope}}{\bar{slope}} \times 100 \%
               =\frac{0.18}{8.37} \times 100 \% = 2.15\% $$

Comparing the calculated data with the experiment data,
$$   \Delta slope = 8.37 - 8.45 = -0.08 s^2/kg        				   $$
$$   \Delta_r slope = \frac{ 8.37 - 8.45}{8.45} \times 100\% = 0.95 \%  $$

For the deviation between theory calculation is not very large, it indicates that the experiment is quite success.

%NOTE: finish here

\subsection{Uncertainty in the $v_{\max}^2\ vs.\ A^2$ relation}

For analyzing the uncertainty of $\Delta x$,
First, investigate the uncertainty of $x_{in}$ and $x_{out}$. 
The uncertainty of type-B of a calliper is 

$$\Delta_{x,B} = \Delta{dev} = 0.02 \times 10^m   $$ 

Take the mean value of the three time measurements of the distance.
To determine type-A uncertainty, the standard deviation of the average value is 

$$   s_{\overline{x_{in}}} = \sqrt{\frac{1}{n(n-1)}\sum_{i=1}^n(x_{in,i}-\overline{x_{in}})^2} $$
$$   s_{\overline{x_{in}}} = 5.3 \times 10^{-3} m 		$$

$$  u_{x_{in}}=\sqrt{\Delta_{x_{in},A}^2+\Delta_{x_{in},B}^2}
     = \sqrt{(0.02 \times 10^{-3} )^2
	 +(5.3 \times 10^{-3})^2} 
	 = 5.3 \times 10^{-3} m $$
$$  u_{r,x_{in}}=\frac{u_{x_{in}}}{\overline{x_{in}}}\times 100\%=1.8\%$$

Similarly, we can calculate $x_{out}$ that
\[
\begin{split}
    &\Delta_{x_{out},A}=t_{0.95}\cdot s_{\overline{x_{out}}}=0.0287\times10^{-3}m\\
    &\Delta_{x,B}=0.02\times10^{-3}m\\
    &u_{x_{out}}=0.03\times10^{-3}m,\quad
    u_{r,x_{out}}=0.2 \%    \\
    &x_{out}=(15.41\pm 0.03) \times10^{-3}m,\quad u_{r,x_{out}}=0.2\%.
\end{split}
\]

Then we can calculate the propagated uncertainty of $\Delta x$

$$\frac{\partial\Delta x}{\partial x_{in}}=\frac{\partial\Delta x}{\partial x_{out}}=\frac{1}{2}  $$

\[
\begin{split}
    u_{\Delta x}&=\sqrt{(\frac{\partial\Delta x}{\partial x_{in}})^2(u_{x_{in}})^2
	+(\frac{\partial\Delta x}{\partial x_{out}})^2(u_{x_{out}})^2}\\ 
    &=0.04\times10^{-3}m,\\
\end{split}
\]

$$    u_{r,\Delta x}=\frac{u_{\Delta x}}{\Delta x}\times100\%=0.4\% $$
$$     \Delta x=(9.95\pm0.04)\times10^{-3}m,\quad u_{r,\Delta x}=0.4\%  $$

For the Uncertainty of the maximum speed $v_{\max}$
Then we can calculate the propagated uncertainty of $v_{\max}=\Delta x/\Delta t$. The partial derivatives are
\[
\begin{split}
    \frac{\partial v_{\max}}{\partial \Delta x}&=\frac{1}{\Delta t}.\\[0.5cm]
    \frac{\partial v_{\max}}{\partial \Delta t}&= - \frac{\Delta x}{(\Delta t)^2}.    
\end{split}    
\]

\[
\begin{split}
    u_{v_{\max}}&=\sqrt{(\frac{\partial v_{\max}}{\partial \Delta x})^2(u_{\Delta x})^2+(\frac{\partial v_{\max}}{\partial \Delta t})^2(u_{\Delta t})^2}     \\
    &=\sqrt{(\frac{1}{\Delta t})^2(u_{\Delta x})^2+(\frac{\Delta x}{(\Delta t)^2})^2(u_{\Delta t})^2}
\end{split}
\]
