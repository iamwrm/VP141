\section{Conclusion and Discussion}

In this lab, we use Stokes’ method to measure fluid viscosity.
The liquid been chosen to measure the viscosity is  the castor oil in this
experiment.

We measures Distance $S$, Time $t$, Diameter of the ball $d$, Inner diameter of
the flask $D$, Density of one ball $\rho_2$, then calculate the viscosity
coefficient not directly.

The relative uncertainty of each physical quantity is quite small, thus our
measurement is very precise and the result is acceptable.

The reason for the accuracy is the nice design of the experiment and carefulness
throughout the experiment.
There are many systematical and random errors leading to the main uncertainty,
but a lot of methods are used to reduce them.

% NOTE: bonus Q
We use 40 balls to calculate the average mass of the ball.
For each ball's mass is quite small and very hard to measure due to the
uncertainty of equipment. We choose 40 instead of 10 to maintain 4 significant
numbers.

% NOTE: bonus Q
Actually the temperature has influence on $\mu$.
The higher temperature is, the more likely the molecule will move, making the
viscosity lower. 

My experiment setup is closer two the air conditioner than other people, so my
result may be slightly affect by the temperature. 

% NOTE: theoretical 
From a engineering website

\url{http://www.engineeringtoolbox.com/absolute-viscosity-liquids-d_1259.html} 
We know that the theoretical value of the viscosity of caster oil is  $0.650
Pa/s$, but the experiment result is $0.7307Pa/s$.
$$ u_{\mu,r} = (0.7307 - 0.650)/ 0.650 \times 100\%  =  12.42 \%   $$ 
The reason is the AC mentioned before cooling down the temperature of my
experiment setup.
We can see here lower temperature results in higher viscosity.


% NOTE: concerns
Furthermore, there are more problems that may cause uncertainty and error in this lab.
\begin{enumerate}
\item The distance between two laser device may not be measured precisely for
  the thread is not strictly perpendicular to the level. 
\item When measuring the time, human's reponse may cause some delay between
  seeing the light fading and pressing down the button on the watch. 
\end{enumerate}

Some ways to improve this lab
\begin{enumerate}
\item Adjust the experimental setup carefully to the level.
\item Throw the ball in the center of the cylinder and in the same position
  every time.
\item Try hard to reduce response time.
\end{enumerate}