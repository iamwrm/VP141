\section{Conclusion and Discussion}

In this lab, we use Stokes’ method to measure fluid viscosity.
The liquid been choosen to measure the viscosity is  the castor oil in this
experiment.

We measures Distance $S$, Time $t$, Diameter of the ball $d$, Inner diameter of
the flask $D$, Density of one ball $\rho_2$ , then calculate the viscosity
coefficient undirectly.

The relative uncertainty of each physical quantity is quite small, thus our
measurement is very precise and the result is acceptable.

The reason for the accuracy is the nice design of the experiment and carefulness
thoroughout the experiment.
There are many systematical and random errors leading to the main uncertainty ,
but a lot of methods are used to reduce them.

% NOTE: bonus Q
We use 40 balls to calculate the average mass of the ball.
For each ball's mass is quite small and very hard to measure due to the uncertainty of equipment.
We choose 40 instead of 10 to maintain 4 significant numbers.

% NOTE: bonus Q
Actually the temperature has influence on $\mu$.
The higher temperature is , the more likely the molecule will move, making the
viscosity lower. 

My experiment setup is closer two the air conditioner than other people, so my
result may be slightly affect by the temperature. 



