\section{Conclusion and Discussion}

In this lab, we use Stokes’ method to measure fluid viscosity.
The liquid been chosen to measure the viscosity is  the castor oil in this
experiment.

We measures Distance $S$, Time $t$, Diameter of the ball $d$, Inner diameter of
the flask $D$, Density of one ball $\rho_2$, then calculate the viscosity
coefficient not directly.

The relative uncertainty of each physical quantity is quite small, thus our
measurement is very precise and the result is acceptable.

The reason for the accuracy is the nice design of the experiment and carefulness
throughout the experiment.
There are many systematical and random errors leading to the main uncertainty,
but a lot of methods are used to reduce them.

% NOTE: bonus Q
We use 40 balls to calculate the average mass of the ball.
For each ball's mass is quite small and very hard to measure due to the
uncertainty of equipment. We choose 40 instead of 10 to maintain 4 significant
numbers.

% NOTE: bonus Q
Actually the temperature has influence on $\mu$.
The higher temperature is, the more likely the molecule will move, making the
viscosity lower. 

My experiment setup is closer two the air conditioner than other people, so my
result may be slightly affect by the temperature. 

% NOTE: theoretical 
From a engineering website
\url{http://www.engineeringtoolbox.com/absolute-viscosity-liquids-d_1259.html} 
We know that the theoretical value of the viscosity of caster oil is  $0.650
Pa/s$, but the experiment result is $0.7307Pa/s$.
$$ u_{\mu,r} = (0.7307 - 0.650)/ 0.650 \times 100\%  =  12.42 \%   $$ 
The reason is the AC mentioned before cooling down the temperature of my
experiment setup.
We can see here lower temperature results in higher viscosity.


% NOTE: concerns

Furthermore, there are more problems that may cause uncertainty and error in this lab.
\begin{enumerate}
1. When we measure the distance between two laser device, the thread cannot be perpendicular to the level.
2. When we measure the time, the trajectory of the balls falling down is not the same.
3. Some balls cannot be recorded because we cannot see very clearly when the ball go through the laser.
4. When we record the time when the ball goes through the laser, there is a
small delay between we see the fact and press down the button on the stopwatch. 
\end{enumerate}

Here are some suggestions that may improve this lab:
\begin{enumerate}
1. Adjust the experimental setup carefully to reach the level. 2. Throw the ball in the same position.
3. Throw the ball in the center line of the cylinder.
4. Be more sensitive to start the record on time. 
Lay a receiver looking toward the laser launcher. When the ball goes through the
first laser, the first receiver will not receive laser and record the time by
computer automatically. When the ball goes through the second laser, the second
receiver will record the time. By subtraction of two time, we can get the time
we want automatically and more accurately than recorded by human eyes. 
\end{enumerate}