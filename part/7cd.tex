\section{Conclusion and Discussion}


In this lab, we have studied the harmonic oscillation 
and the relations between $T$, $M$, $k$ and $v_{\max}$. 
Now we analyze the results.


\subsection{Spring Constant}


\begin{table}[H]
\centering
\begin{tabular}{|c|c|}
spring 1 & 2.3311 $\pm$ 0.013[N/m] \\ \hline
spring 2 & 2.3206 $\pm$ 0.0105[N/m] \\ \hline
spring series & 1.165 $\pm$ 0.0380[N/m]  \\ \hline
\end{tabular}
\caption{The spring constant}
\end{table}


For the relative uncertainty are all very small,
the experiment in this part is very accurate.

By theory, we can calculate $k_3$, i.e. the $k$ of the spring serial by 
$$ k_{3,theory} = \frac{k_1 \cdot k_2 }{k_1 + k_2} =  1.1629 $$ 

Compared with $k_3 = 1.1649$  from the experiment,
$$ u_{k_{3,theory},k_3} = \frac{1.1649 - 1.1629}{1.1629} \cdot 100 \%  = 0.17 \% $$ 
The theory data is close to the experiment data.

The accurate experimental results prove that Hooke’s Law.



\subsection{Relation between the period $T$ and the mass $M$}

