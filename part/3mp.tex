\section{Measurement Procedue}

\setenumerate[2]{label= (\arabic*)}

\subsection{Spring constant}

\begin{enumerate}
\item Adjust the Jolly balance to vertical and attach the spring Add a 20g
  preload and adjust knob $I_1$ and $I_2$ to make sure the mirror can move
  freely through the tube. Check if the balance parallel to the spring. 
\item Adjust knob $G$ to make the three lines in the tube coincide.
\item Reading the reading on the scale, add mass from 1 to 6 and record $L_i$ in
  order. 
\item Estimate the spring constant $k_1$ by fitting.
\item Repeat the measurements with spring 2. Calculate $k_2$.
\item Remove the preload and repeat the measurements for spring 1 and 2
  connected. Calculate $k_3$. 
\end{enumerate}

\subsection{Relation between oscillation period $T$ and the mass of the oscillator $M$} 

\begin{enumerate}
\item Adjust the air track so that it's horizontal.

\begin{enumerate}
\item Turn on the air track and check whether there's blocked holes.
\item Place the cart on the track. Adjust the track with the knob on the side
  with a single one. 
\end{enumerate}

\item Horizontal air track

\begin{enumerate}
\item Attach springs to the sides of the cart and set up the I-shape shutter.
  Make sure that the photoelectric gate is at the equilibrium position. 
\item Set the timer into ``T'' mode. Add weight in order and release it with a
  caliper. Record the corresponding time intervals for 10 periods. 
\item Analyze the relation between $M$ and $T$ by plotting a graph. 
\end{enumerate}

\item Inclined air track

\begin{enumerate}
\item Add three plastic plates under the air track every time. Repeat the steps
  in 2. 
\item Analyze the relation between $M$ and $T$ by plotting a graph.
\end{enumerate}

\end{enumerate}

\subsection{Relation between the oscillation period and the amplitude} 

\begin{enumerate}

\item Keep the mass of the cart unchanged and change the amplitude (choose 6
  different values). The amplitude is about 5.0/ 10.0/ 15.0/ ... /30.0 cm. 
\item Apply linear fit to the data and comment on the relation between the
  oscillation period $T$ and the amplitude $A$ based on the correlation
  coefficient $\gamma$. 
\end{enumerate}

\subsection{Relation between the maximum speed and the amplitude}

\begin{enumerate}
\item Keep the mass of the cart unchanged and change the amplitude from $5.0\
  to\ 30.0 cm$ with gap of $5.0cm$. 
\item Measure $v_{\max}$ for different amplitudes.
\end{enumerate}

\subsection{Mass measurement}

\begin{enumerate}
\item Adjust the electronic balance every time before you use it. The level
  bubble should be in the center of the circle. 
\item Add weights according to a fixed order. Weigh the cart with the I-shape
  shutter and with the U-shape shutter. Measure the mass of spring 1 and spring
  2. 
\item Record the data only after the circular symbol on the scales display
  disappears. 
\end{enumerate}


\section{Caution}

\begin{itemize}
\item Do not stretch the spring over its elastic limit, otherwise the spring
  will not return to its original shape.  
\item When using the Jolly balance, the mirror should be moving freely in the
  glass tube without any friction. When adding weight, hold the tray steady to
  avoid errors due to vibrations. 
\item Please use tweezers to move the weights around.
\item  Make sure that no air holes on the air tracks are blocked.
\item Avoid scratching the cart. Do not move the cart when pressed against the
  air track.
\end{itemize}
