\section{Measurement Procedue}

\setenumerate[2]{label=(\arabic*)}

\begin{enumerate}
\item Adjustment of the Stokes’ viscosity measurement device.

\begin{enumerate}
\item Adjust the knobs beneath the base to make the plumb aiming at the center of the base.
\item Turn on the two lasers, adjust the beams so that they are parallel and aim at the plumb line.
\item Remove the plumb and place the graduated flask with castor oil at the center of the base.
\item Place the guiding pipe on the top of the viscosity measurement device.
\item Put a metal ball into the pipe and check whether the ball, falling down in
  the oil, can blocks the laser beams. If not, repeat Step 1.
\end{enumerate}

\item Measurement of the (constant) velocity of a falling ball.

\begin{enumerate}
\item Measure the vertical distance s between the two laser beams at least three
  times. 
\item Put a metal ball into the guiding pipe. Start the stopwatch when the ball 
  passes through the first beam, and stop it when it passes through the second
  one. Record the time t and repeat the procedure for at least six times. 
\end{enumerate}

\item Measurement of the ball density ρ2.

\begin{enumerate}
\item Use electronic scales to measure the mass of 40 metal balls. Calculate the
  average to find the mass of a single ball. 
\item Use a micrometer to measure the diameter of the metal balls. Repeat for
  ten times and calculate the average value. 
\item Calculate the ball density ρ2.
\end{enumerate}

\item Measure of the density ρ1 of the castor oil by using the provided
  densimeter (one measurement). Use a calliper to measure the inner diameter D
  of the graduated flask for six times. Read the ambient temperature from the
  thermometer placed in the lab. 
\item Calculate the value of viscosity coefficient η using Eq. (5).

\end{enumerate}

