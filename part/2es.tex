\section{Measurement Method}

    The experimental setup consists of a signal source, two piezoelectric transducers $S_1$ and $S_2$ , and oscilloscope arranged as shown in Figure \ref{app}.
    \begin{figure}[H]
        \centering
        \includegraphics[height=5cm]{fig/es1}
        \caption{Experimental setup}\label{app}
    \end{figure}


\subsubsection{Resonance method}
    The elements $S_1$ and $S_2$ are the wave source and the receiver (also reflector), respectively, placed a distance $L$ from each other. If they are arranged parallel to each other, the sound wave is reflected. If
    \begin{equation}\label{equ_res}
        L=n\frac{\lambda}{2},
    \end{equation}
    where $n=1,2,\cdots,$ \emph{i.e.} the distance is a multiple of half-wavelength, standing waves will form, and maximum output power will be observed in the oscillograph (Figure 2). The distance between two successive maxima ($L_{i+1} - L_i$) is always $\lambda/2$. After the position corresponding to each maximum is measured, it is easy to find the wavelength and then the speed of sound by using Eq. \ref{vlf}. The frequency $f$ is displayed directly on the signal generator.
    \begin{figure}[h]
        \centering
        \includegraphics[height=5cm]{fig/r}
        \caption{Relation between the signal voltage and the distance}\label{vL}
    \end{figure}

\subsubsection{Phase-comparison method}

    If the phase of the wave at two points on the wave propagation direction is equal, then the distance between these points L has to be a multiple of the wavelength, \emph{i.e.}
    \[
        L=n\lambda,
    \]
    where $n=1,2,\cdots$. The experimental setup for the phase comparison method is the same as in the previous method (Figure \ref{app}). Lissajous figures are used to identify the values of L. Lissajous figures (or Lissajous curves) are trajectories of a particle that moves in a plane so that \emph{i.e.} it moves in a harmonic motion independently along two perpendicular directions (for example the axes x and y of a Cartesian coordinate system), so that $\textbf{r}(t) = (A_x cos(\omega_x t + \phi_x ),\ A_y cos(\omega_y t + \phi_y ))$. When the two superimposed harmonic motions have identical frequency $\omega x = \omega y$ and phase difference $|\phi x - \phi y | = n\pi$, where $n = 0, 1, 2,\cdots$, the Lissajous figure will show as a straight line. For other values of the phase difference the figures will have an elliptical shape.\\

\subsubsection{Time-difference method}
    When an ultrasonic pulse signal emitted by $S_1$ arrives at $S_2$, it is received and returned back to the processor. By contrasting the original signal with the received one, one can measure the time needed for the sound to travel from $S_1$ to $S_2$ over a distance of L. When the values of $L$ and $t$ are known, the phase speed of sound can be found from Eq. \ref{vlt}.\\
    
\subsubsection{Successive difference method}
    The successive difference method is an effective method to increase the accuracy of the average value calculated from a series of measurement data. In this experiment, the usual method of calculating the average value, illustrated by the formula
    \begin{equation}\label{equ_suc}
        \frac{\bar{\lambda}}{2}=\frac{[(L_1-L_0)+(L_2-L_1)+\cdots+(L_n-L_{n-1})]}{n}=\frac{L_n-L_0}{n},
    \end{equation}
    will be modified, because as Eq. \ref{equ_suc} shows, the average value of the wavelength is determined only by the first and the last value, $L_0$ and $L_n$ .

    A modification of the formula by rearranging terms as
    \begin{equation}\label{equ_suc}
        n\frac{\bar{\lambda}}{2}=\frac{\Sigma_{i=1}^{n}(L_{n+i}-L_i)}{n},
    \end{equation}
    produces more accurate results, as each value contributes to the final result.