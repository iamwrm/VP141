\section{Result}

\subsection{Measurements for spring constant}

First we get the raw length measurement data in Table \ref{s1}.
\begin{table}[H]
\centering
\begin{tabular}{|p{0.7cm}|p{3cm}||p{0.7cm}|p{3cm}||p{0.7cm}|p{3cm}|}
\hline
\multicolumn{2}{|c|}{spring 1 [cm] $\pm$ 0.01 [cm]} &
\multicolumn{2}{|c|}{spring 2 [cm] $\pm$ 0.01 [cm]} &
\multicolumn{2}{|c|}{serial   [cm] $\pm$ 0.01 [cm]} \\ \hline
$L_0$ & 0.00  & $L_0$ & 0.00  & $L_0$ & 20.00 \\ \hline
$L_1$ & 2.00  & $L_1$ & 2.01  & $L_1$ & 23.96 \\ \hline
$L_2$ & 3.96  & $L_2$ & 4.00  & $L_2$ & 27.94 \\ \hline
$L_3$ & 5.96  & $L_3$ & 6.03  & $L_3$ & 31.90 \\ \hline
$L_4$ & 8.02  & $L_4$ & 8.06  & $L_4$ & 36.40 \\ \hline
$L_5$ & 10.04 & $L_5$ & 10.08 & $L_5$ & 39.94 \\ \hline
$L_6$ & 12.02 & $L_6$ & 12.10 & $L_6$ & 44.02 \\ \hline
\end{tabular}
\caption{Spring constant measurement data}
\label{s1}
\end{table}

Then we can calculate the change amount of the spring length by $\Delta L_i=L_i-L_0$ and get Table \ref{s2}.

\begin{table}[H]
\centering
\begin{tabular}{|p{0.7cm}|p{3cm}||p{0.7cm}|p{3cm}||p{0.7cm}|p{3cm}|}
\hline
\multicolumn{2}{|c|}{spring 1 [cm] $\pm$ 0.01 [cm]} &
\multicolumn{2}{|c|}{spring 2 [cm] $\pm$ 0.01 [cm]} &
\multicolumn{2}{|c|}{serial   [cm] $\pm$ 0.01 [cm]} \\ \hline
$\Delta L_1$ & 2.00  & $\Delta L_1$ & 2.01  & $\Delta L_1$ & 3.96  \\ \hline
$\Delta L_2$ & 3.96  & $\Delta L_2$ & 4.00  & $\Delta L_2$ & 7.94  \\ \hline
$\Delta L_3$ & 5.96  & $\Delta L_3$ & 6.03  & $\Delta L_3$ & 11.90 \\ \hline
$\Delta L_4$ & 8.02  & $\Delta L_4$ & 8.06  & $\Delta L_4$ & 16.40 \\ \hline
$\Delta L_5$ & 10.04 & $\Delta L_5$ & 10.08 & $\Delta L_5$ & 19.94 \\ \hline
$\Delta L_6$ & 12.02 & $\Delta L_6$ & 12.10 & $\Delta L_6$ & 24.02 \\ \hline
\end{tabular}
\caption{Calculated Spring constant measurement data}
\label{s2}
\end{table}

We get the mass of the weight object for every $\Delta L_i$ in Table
\ref{massofweight}.

Since the acceleration due to gravity in Shanghai is $9.794 m/s^2$, we calculated
the weights of each weight object from its mass, as shown in Table
\ref{gravityofweight}. 

\begin{minipage}{0.5\linewidth}
\begin{table}[H]
\centering
\begin{tabular}{|c|c|}
\hline
\multicolumn{2}{|c|}{m [g] $\pm$ 0.01 [g]} \\ \hline
1 & 4.65  \\ \hline
2 & 9.32  \\ \hline
3 & 14.17 \\ \hline
4 & 18.99 \\ \hline
5 & 23.80 \\ \hline
6 & 28.51 \\ \hline
\end{tabular}
\caption{Mass measurement data.}
\label{massofweight}
\end{table}
\end{minipage}
%
\begin{minipage}{0.5\linewidth}
\begin{table}[H]
\centering
\begin{tabular}{|c|c|}
\hline
\multicolumn{2}{|c|}{F [N] $\pm$ 0.0001 [N]} \\ \hline
1  & 0.0455  \\ \hline 
2  & 0.0913  \\ \hline 
3  & 0.1388  \\ \hline 
4  & 0.1860  \\ \hline 
5  & 0.2331  \\ \hline 
6  & 0.2792  \\ \hline 
\end{tabular}
\caption{Weight measurement data.}
\label{gravityofweight}
\end{table}
\end{minipage}

For Spring 1, we can have its length change data versus the force affected on it
data. 

\begin{table}[H]
\centering
\begin{tabular}{|c|c|c|}
\hline
No. & length [cm] $\pm$ 0.01 [cm] & F [N] $\pm$ 0.0001 [N] \\ \hline
$\Delta L_1$ & 2.00  &  0.0455  \\ \hline
$\Delta L_2$ & 3.96  &  0.0913  \\ \hline
$\Delta L_3$ & 5.96  &  0.1388  \\ \hline
$\Delta L_4$ & 8.02  &  0.1860  \\ \hline
$\Delta L_5$ & 10.04 &  0.2331  \\ \hline
$\Delta L_6$ & 12.02 &  0.2792  \\ \hline
\end{tabular}
\caption{$\Delta L$  vs. Force}
\label{s1df}
\end{table}


