\subsection{Uncertainty of the slope of $T^2\ vs.\ M$}
    The uncertainty of one period is $u_T/10=1\times10^{-5}s$ and the uncertainty of mass is
    \[
    \begin{split}
        M&=m_{objI}+\frac{1}{3}m_{spr1}+\frac{1}{3}m_{spr2}+m_i,\\
        u_M&=\sqrt{
         (\frac{\partial M}{\partial m_{objI}})^2(u_{m_{objI}})^2
        +(\frac{\partial M}{\partial m_{spr1}})^2(u_{m_{spr1}})^2
        +(\frac{\partial M}{\partial m_{spr2}})^2(u_{m_{spr2}})^2
        +(\frac{\partial M}{\partial m_i})^2(u_{m_i})^2
        }\\
        &=\sqrt{
         (1)^2(u_{m_{objI}})^2
        +(\frac{1}{3})^2(u_{m_{spr1}})^2
        +(\frac{1}{3})^2(u_{m_{spr2}})^2
        +(1)^2(u_{m_i})^2
        }\\
        &=\sqrt{(u_{m_{objI}})^2+(u_{m_{spr1}})^2/9+(u_{m_{spr2}})^2/9+(u_{m_i})^2}.
    \end{split}
    \]

    when data provided 
	$m_{objI}=(176.55\pm0.01)\times10^{-3}kg$, $m_{spr1}=(10.74\pm0.01)\times10^{-3}kg$, $m_{spr2}=(10.77\pm0.01)\times10^{-3}kg$ and $m_1=(4.83\pm0.01)\times10^{-3}kg$,
    \[
    \begin{split}
        u_M & = \sqrt{0.00001^2+0.00001^2/9+0.00001^2/9+0.00001^2}=0.0000149\\
            & = 1.5\times10^{-5}kg.
    \end{split}
    \]

    For $T^2$,
    \[
    \begin{split}
        &\frac{\partial T^2}{\partial T}=2T,\\
        &u_{T^2}=\sqrt{(2T)^2(u_T)^2}=2Tu_T.
    \end{split}
    \]


    From the formula of period $T=2\pi\sqrt{\frac{M}{k}}$ we know that the slope of $T^2\ vs.\ M$ is
    \begin{equation}\label{slope}
        slope=\frac{T^2}{M}=\frac{4\pi^2}{k},
    \end{equation}
    where k is the effective spring constant. In this experiment, the effective spring constant can be calculated by
    \[ 
    \begin{split}
        F&=k_1\Delta x+k_2\Delta x,\\
        F&=k_{eff}\Delta x.
    \end{split}
    \]
    Thus we get $k_{eff}=k_1+k_2$. We take the reults of $k_1$ and $k_2$ as the theoretical value and ignore its uncertainty, then
    \begin{equation}\label{keff}
        k_{eff}=k_1+k_2=2.22+2.25=4.47kg/s^2.
    \end{equation}
    Plugging it into Eq. \ref{slope}, we get the theoretical slope $slope_t=4\times 3.14^2/4.47=8.82s^2/kg$
    By curve fitting, we get the slopes
    \[
    \begin{split}
        slope_{hor}&=8.72\pm 0.05s^2/kg,\quad u_{r,hor}=0.6\%,\\
        slope_{inc1}&=8.73\pm 0.14s^2/kg,\quad u_{r,inc1}=1.6\%,\\
        slope_{inc2}&=8.76\pm 0.17s^2/kg,\quad u_{r,inc2}=1.9\%,
    \end{split}
    \]
    where the uncertainty is calculated by $t_0.95/\sqrt{6-2}\times \sigma$.
    Hence the experimental value of the slope is $slope_e=\overline{slope}=8.74s^/kg$.
    Similarly, we get the propagated and relative uncertainty of slope, which is
    \[
    \begin{split}
        u_{slope}&=\sqrt{(\frac{\partial slope}{\partial slope_{hor}})^2\cdot(u_{slope_{hor}})^2+(\frac{\partial slope}{\partial slope_{inc1}})^2\cdot(u_{slope_{inc1}})^2+(\frac{\partial slope}{\partial slope_{inc2}})^2\cdot(u_{slope_{inc2}})^2}\\
        &=\sqrt{(\frac{1}{3})^2\cdot(0.05)^2+(\frac{1}{3})^2\cdot(0.14)^2+(\frac{1}{3})^2\cdot(0.17)^2}=0.08s^2/kg.\\
        u_{r,slope}&=\frac{u_{slope}}{\overline{slope}}\times100\%=\frac{0.08}{8.74}\times100\%=0.9\%.
    \end{split}
    \]

    Hence, $slope_e=\overline{slope}=8.74\pm 0.08s^/kg, \quad u_{r,slope}=0.9\%$. Compared with the "theoretical" value of $slope_t=8.82s^2/kg$, we calculate the deviation $\Delta slope$ and relative deviation $\Delta_r slope$
    \[
    \begin{split}
        \Delta slope&=8.74-8.82=-0.08s^2/kg,\\
        \Delta_r slope&=\frac{8.74-8.82}{8.82}\times 100\%=-0.9\%.
    \end{split}
    \]
    The theoretical value is just on the boundary of uncertainty interval. It shows that our experimental values are relatively reliable.

\subsection{Uncertainty in the $v_{max}^2\ vs.\ A^2$ relation}
\subsubsection{Uncertainty of $\Delta x$}
    First, we need to determine the uncertainty of $x_{in}$ and $x_{out}$. The uncertainty of type-B of a calliper is $\Delta_{x,B}=\Delta{dev}=0.02\times10^{-3}m$. The distance is found by the average value of 3 measurements. To estimate  type-A uncertainty, the standard deviation of the average value is 
    \[
        s_{\overline{x_{in}}}=\sqrt{\frac{1}{n(n-1)}\sum_{i=1}^n(x_{in,i}-\overline{x_{in}})^2}.
    \]
    Using the data from Table \ref{x} we find that $s_{\overline{x_{in}}}\approx 0.0176\times10^{-3}m$. Considering $t_{0.95}=4.30$ for $n=3$, the type-A uncertainty is estimated as $\Delta_{x_{in},A}=4.30\times0.0176\times10^{-3}m\approx 0.0757\times10^{-3}m$.\\
    Hence the combined uncertainty is
    \[
         u_{x_{in}}=\sqrt{\Delta_{x_{in},A}^2+\Delta_{x_{in},B}^2}=\sqrt{(0.02\times10^{-3})^2+(0.0757\times10^{-3})^2}\approx 0.08\times10^{-3}m
    \]
     and the corresponding relative uncertainty is 
    \[
        u_{r,x_{in}}=\frac{u_{x_{in}}}{\overline{x_{in}}}\times 100\%=1.8\%.
    \]
    The experimentally found $x_{in}$ is 
    \[
        x_{in}=(4.49\pm 0.08) \times10^{-3}m,\quad u_{r,x_{in}}=1.8\%.
    \]
    Similarly, we know about $x_{out}$ that
    \[
    \begin{split}
        &\Delta_{x_{out},A}=t_{0.95}\cdot s_{\overline{x_{out}}}=0.0287\times10^{-3}m\\
        &\Delta_{x,B}=0.02\times10^{-3}m\\
        &u_{x_{out}}=0.03\times10^{-3}m,\quad
        u_{r,x_{out}}=0.2\%\\[0.4cm]
        &x_{out}=(15.41\pm 0.03) \times10^{-3}m,\quad u_{r,x_{out}}=0.2\%.
    \end{split}
    \]
    Then we can calculate the propagated uncertainty of $\Delta x$
    \[
    \begin{split}
        \frac{\partial\Delta x}{\partial x_{in}}&=\frac{\partial\Delta x}{\partial x_{out}}=\frac{1}{2},\\
        u_{\Delta x}&=\sqrt{(\frac{\partial\Delta x}{\partial x_{in}})^2(u_{x_{in}})^2+(\frac{\partial\Delta x}{\partial x_{out}})^2(u_{x_{out}})^2}\\
        &=\sqrt{(\frac{1}{2})^2(0.00008)^2+(\frac{!}{2})^2(0.00003)^2}\\
        &=0.04\times10^{-3}m,\\
        u_{r,\Delta x}&=\frac{u_{\Delta x}}{\Delta x}\times100\%=0.4\%\\[0.4cm]
        \Delta x&=(9.95\pm0.04)\times10^{-3}m,\quad u_{r,\Delta x}=0.4\%.
    \end{split}
    \]
\subsubsection{Uncertainty of the maximum speed $v_{max}$}
    Then we can calculate the propagated uncertainty of $v_{max}=\Delta x/\Delta t$. The partial derivatives are
    \[
    \begin{split}
        \frac{\partial v_{max}}{\partial \Delta x}&=\frac{1}{\Delta t}.\\[0.5cm]
        \frac{\partial v_{max}}{\partial \Delta t}&=-\frac{\Delta x}{(\Delta t)^2}.    
    \end{split}    
    \]
    Hence,
    \[
    \begin{split}
        u_{v_{max}}&=\sqrt{(\frac{\partial v_{max}}{\partial \Delta x})^2(u_{\Delta x})^2+(\frac{\partial v_{max}}{\partial \Delta t})^2(u_{\Delta t})^2}\\[0.4cm]
        &=\sqrt{(\frac{1}{\Delta t})^2(u_{\Delta x})^2+(-\frac{\Delta x}{(\Delta t)^2})^2(u_{\Delta t})^2}
    \end{split}
    \]
