\subsection{Uncertainty of the slope of $T^2\ vs.\ M$}
    The uncertainty of one period is $u_T/10=1\times10^{-5}s$ and the uncertainty of mass is
    \[
    \begin{split}
        M&=m_{objI}+\frac{1}{3}m_{spr1}+\frac{1}{3}m_{spr2}+m_i,\\
        u_M&=\sqrt{
         (\frac{\partial M}{\partial m_{objI}})^2(u_{m_{objI}})^2
        +(\frac{\partial M}{\partial m_{spr1}})^2(u_{m_{spr1}})^2
        +(\frac{\partial M}{\partial m_{spr2}})^2(u_{m_{spr2}})^2
        +(\frac{\partial M}{\partial m_i})^2(u_{m_i})^2
        }\\
        &=\sqrt{
         (1)^2(u_{m_{objI}})^2
        +(\frac{1}{3})^2(u_{m_{spr1}})^2
        +(\frac{1}{3})^2(u_{m_{spr2}})^2
        +(1)^2(u_{m_i})^2
        }\\
        &=\sqrt{(u_{m_{objI}})^2+(u_{m_{spr1}})^2/9+(u_{m_{spr2}})^2/9+(u_{m_i})^2}.
    \end{split}
    \]

    \[
    \begin{split}
        u_M&=\sqrt{0.00001^2+0.00001^2/9+0.00001^2/9+0.00001^2}=0.0000149\\
        &\approx 1.5\times10^{-5}kg.
    \end{split}
    \]

    The uncertainty of $T^2$ is propagated uncertainty, which is calculated as
    \[
    \begin{split}
        &\frac{\partial T^2}{\partial T}=2T,\\
        &u_{T^2}=\sqrt{(2T)^2(u_T)^2}=2Tu_T.
    \end{split}
    \]

    \textbf{For example}, for $T=1.27989s$, $u_{T^2}=2\times1.27989\times1\times10^{-5}=2.55978\times10^{-5}\approx3\times10^{-5}s^2$. The complete results are listed in Table \ref{tmdata}, \ref{tmi1data} and \ref{tmi2data}.

    From the formula of period $T=2\pi\sqrt{\frac{M}{k}}$ we know that the slope of $T^2\ vs.\ M$ is
    \begin{equation}\label{slope}
        slope=\frac{T^2}{M}=\frac{4\pi^2}{k},
    \end{equation}
    where k is the effective spring constant. In this experiment, the effective spring constant can be calculated by
    \[ 
    \begin{split}
        F&=k_1\Delta x+k_2\Delta x,\\
        F&=k_{eff}\Delta x.
    \end{split}
    \]
    Thus we get $k_{eff}=k_1+k_2$. We take the reults of $k_1$ and $k_2$ as the theoretical value and ignore its uncertainty, then
    \begin{equation}\label{keff}
        k_{eff}=k_1+k_2=2.22+2.25=4.47kg/s^2.
    \end{equation}
    Plugging it into Eq. \ref{slope}, we get the theoretical slope $slope_t=4\times 3.14^2/4.47=8.82s^2/kg$
    By curve fitting, we get the slopes
    \[
    \begin{split}
        slope_{hor}&=8.72\pm 0.05s^2/kg,\quad u_{r,hor}=0.6\%,\\
        slope_{inc1}&=8.73\pm 0.14s^2/kg,\quad u_{r,inc1}=1.6\%,\\
        slope_{inc2}&=8.76\pm 0.17s^2/kg,\quad u_{r,inc2}=1.9\%,
    \end{split}
    \]
    where the uncertainty is calculated by $t_0.95/\sqrt{6-2}\times \sigma$.
    Hence the experimental value of the slope is $slope_e=\overline{slope}=8.74s^/kg$.
    Similarly, we get the propagated and relative uncertainty of slope, which is
    \[
    \begin{split}
        u_{slope}&=\sqrt{(\frac{\partial slope}{\partial slope_{hor}})^2\cdot(u_{slope_{hor}})^2+(\frac{\partial slope}{\partial slope_{inc1}})^2\cdot(u_{slope_{inc1}})^2+(\frac{\partial slope}{\partial slope_{inc2}})^2\cdot(u_{slope_{inc2}})^2}\\
        &=\sqrt{(\frac{1}{3})^2\cdot(0.05)^2+(\frac{1}{3})^2\cdot(0.14)^2+(\frac{1}{3})^2\cdot(0.17)^2}=0.08s^2/kg.\\
        u_{r,slope}&=\frac{u_{slope}}{\overline{slope}}\times100\%=\frac{0.08}{8.74}\times100\%=0.9\%.
    \end{split}
    \]

    Hence, $slope_e=\overline{slope}=8.74\pm 0.08s^/kg, \quad u_{r,slope}=0.9\%$. Compared with the "theoretical" value of $slope_t=8.82s^2/kg$, we calculate the deviation $\Delta slope$ and relative deviation $\Delta_r slope$
    \[
    \begin{split}
        \Delta slope&=8.74-8.82=-0.08s^2/kg,\\
        \Delta_r slope&=\frac{8.74-8.82}{8.82}\times 100\%=-0.9\%.
    \end{split}
    \]
    The theoretical value is just on the boundary of uncertainty interval. It shows that our experimental values are relatively reliable.

\subsection{Uncertainty in the $v_{max}^2\ vs.\ A^2$ relation}
\subsubsection{Uncertainty of $\Delta x$}
    First, we need to determine the uncertainty of $x_{in}$ and $x_{out}$. The uncertainty of type-B of a calliper is $\Delta_{x,B}=\Delta{dev}=0.02\times10^{-3}m$. The distance is found by the average value of 3 measurements. To estimate  type-A uncertainty, the standard deviation of the average value is 
    \[
        s_{\overline{x_{in}}}=\sqrt{\frac{1}{n(n-1)}\sum_{i=1}^n(x_{in,i}-\overline{x_{in}})^2}.
    \]
    Using the data from Table \ref{x} we find that $s_{\overline{x_{in}}}\approx 0.0176\times10^{-3}m$. Considering $t_{0.95}=4.30$ for $n=3$, the type-A uncertainty is estimated as $\Delta_{x_{in},A}=4.30\times0.0176\times10^{-3}m\approx 0.0757\times10^{-3}m$.\\
    Hence the combined uncertainty is
    \[
         u_{x_{in}}=\sqrt{\Delta_{x_{in},A}^2+\Delta_{x_{in},B}^2}=\sqrt{(0.02\times10^{-3})^2+(0.0757\times10^{-3})^2}\approx 0.08\times10^{-3}m
    \]
     and the corresponding relative uncertainty is 
    \[
        u_{r,x_{in}}=\frac{u_{x_{in}}}{\overline{x_{in}}}\times 100\%=1.8\%.
    \]
    The experimentally found $x_{in}$ is 
    \[
        x_{in}=(4.49\pm 0.08) \times10^{-3}m,\quad u_{r,x_{in}}=1.8\%.
    \]
    Similarly, we know about $x_{out}$ that
    \[
    \begin{split}
        &\Delta_{x_{out},A}=t_{0.95}\cdot s_{\overline{x_{out}}}=0.0287\times10^{-3}m\\
        &\Delta_{x,B}=0.02\times10^{-3}m\\
        &u_{x_{out}}=0.03\times10^{-3}m,\quad
        u_{r,x_{out}}=0.2\%\\[0.4cm]
        &x_{out}=(15.41\pm 0.03) \times10^{-3}m,\quad u_{r,x_{out}}=0.2\%.
    \end{split}
    \]
    Then we can calculate the propagated uncertainty of $\Delta x$
    \[
    \begin{split}
        \frac{\partial\Delta x}{\partial x_{in}}&=\frac{\partial\Delta x}{\partial x_{out}}=\frac{1}{2},\\
        u_{\Delta x}&=\sqrt{(\frac{\partial\Delta x}{\partial x_{in}})^2(u_{x_{in}})^2+(\frac{\partial\Delta x}{\partial x_{out}})^2(u_{x_{out}})^2}\\
        &=\sqrt{(\frac{1}{2})^2(0.00008)^2+(\frac{!}{2})^2(0.00003)^2}\\
        &=0.04\times10^{-3}m,\\
        u_{r,\Delta x}&=\frac{u_{\Delta x}}{\Delta x}\times100\%=0.4\%\\[0.4cm]
        \Delta x&=(9.95\pm0.04)\times10^{-3}m,\quad u_{r,\Delta x}=0.4\%.
    \end{split}
    \]
\subsubsection{Uncertainty of the maximum speed $v_{max}$}
    Then we can calculate the propagated uncertainty of $v_{max}=\Delta x/\Delta t$. The partial derivatives are
    \[
    \begin{split}
        \frac{\partial v_{max}}{\partial \Delta x}&=\frac{1}{\Delta t}.\\[0.5cm]
        \frac{\partial v_{max}}{\partial \Delta t}&=-\frac{\Delta x}{(\Delta t)^2}.    
    \end{split}    
    \]
    Hence,
    \[
    \begin{split}
        u_{v_{max}}&=\sqrt{(\frac{\partial v_{max}}{\partial \Delta x})^2(u_{\Delta x})^2+(\frac{\partial v_{max}}{\partial \Delta t})^2(u_{\Delta t})^2}\\[0.4cm]
        &=\sqrt{(\frac{1}{\Delta t})^2(u_{\Delta x})^2+(-\frac{\Delta x}{(\Delta t)^2})^2(u_{\Delta t})^2}
    \end{split}
    \]

    \textbf{For example}, in the case that $\Delta x=0.00995\pm 0.00004m$ and $\Delta t=0.04169\pm 0.00001s$,
    \[
    \begin{split}
        u_{v_{max}}&=\sqrt{(0.00004)^2/(0.04169)^2+(0.00995)^2(0.00001)^2/(0.04169)^4}\approx0.001m/s,\\
        u_{r,v_{max}}&=\frac{u_{v_{max}}}{v_{max}}\times100\%\approx0.4\%
    \end{split}
    \]
    \begin{table}[!h] \small
        \centering
        \begin{tabular}{|c|c|c|c|c|}
            \hline
            No. & $\Delta x[m]$ & $u_{\Delta x}[m]$ & $\Delta t[s]$ & $u_{\Delta t}[s]$\\ \hline
            1 & 0.00995 & 0.00004 & 0.04169 & 0.00001\\ \hline
            2 & 0.00995 & 0.00004 & 0.02061 & 0.00001\\ \hline
            3 & 0.00995 & 0.00004 & 0.01382 & 0.00001\\ \hline
            4 & 0.00995 & 0.00004 & 0.01045 & 0.00001\\ \hline
            5 & 0.00995 & 0.00004 & 0.00839 & 0.00001\\ \hline
            6 & 0.00995 & 0.00004 & 0.00703 & 0.00001\\ \hline
        \end{tabular}
        \caption{Data for the calculation of $v_{max}$}\label{dt}
    \end{table}

    Since the calculations are repeated and too complicated, the results of $v_{max}$ are listed in Table \ref{vadata1}.
    \begin{table}[!h] \small
        \centering
        \begin{tabular}{|c|c|c|c|}
            \hline
            No. & $v_{max}[m/s]$ & $u_{v_{max}}[m/s]$ & $u_{r,v_{max}}[\%]$\\ \hline
            1 & 0.239 & 0.001 & 0.4\\ \hline
            2 & 0.483 & 0.002 & 0.4\\ \hline
            3 & 0.720 & 0.003 & 0.4\\ \hline
            4 & 0.952 & 0.004 & 0.4\\ \hline
            5 & 1.19 & 0.005 & 0.4\\ \hline
            6 & 1.42 & 0.006 & 0.4\\ \hline
        \end{tabular}
        \caption{Results of $v_{max}$}\label{vadata1}
    \end{table}

    Then we can calculate the propagated uncertainty of $v_{max}^2$. The partial derivative is
    \[
        \frac{\partial v_{max}^2}{\partial v_{max}}=2v_{max}.
    \]
    Hence,
    \[
        u_{v_{max}^2}=\sqrt{(\frac{\partial v_{max}^2}{\partial v_{max}})^2(u_{v_{max}}})^2=2v_{max}u_{v_{max}}\\[0.4cm]
    \]

    \textbf{For example}, in the case that $v_{max}=0.239\pm 0.001m/s$,
    \[
    \begin{split}
        u_{v_{max}^2}&=2\times0.239\times0.001\approx0.0005m/s,\\
        u_{r,v_{max}^2}&=\frac{u_{v_{max}^2}}{v_{max}^2}\times100\%\approx0.8\%
    \end{split}
    \]
    The results of $v_{max}^2$ are shown in Table \ref{vadata2}.
    \begin{table}[!h] \small
        \centering
        \begin{tabular}{|c|c|c|c|}
            \hline
            No. & $v_{max}^2[m^2/s^2]$ & $u_{v_{max}^2}[m^2/s^2]$ & $u_{r,v_{max}^2}[\%]$\\ \hline
            1 & 0.0570 & 0.0005 & 0.8\\ \hline
            2 & 0.233 & 0.002 & 0.8\\ \hline
            3 & 0.518 & 0.004 & 0.8\\ \hline
            4 & 0.907 & 0.007 & 0.8\\ \hline
            5 & 1.41 & 0.012 & 0.8\\ \hline
            6 & 2.00 & 0.017 & 0.9\\ \hline
        \end{tabular}
        \caption{Results of $v_{max}$}\label{vadata2}
    \end{table}

    Now we will calculate the square of amplitudes $A^2$. The uncertainty of $A$ is $u_A=\Delta_{dev}=0.001m$.
    \[
    \begin{split}
        &\frac{\partial A^2}{\partial A}=2A.\\
        &u_{A^2}=\sqrt{(\frac{\partial A^2}{\partial A})^2(u_A)^2}=2Au_A\\[0.4cm]
    \end{split}
    \]

    \textbf{For example}, in the case that $A=0.050\pm 0.001m$,
    \[
    \begin{split}
        u_{A^2}&=2\times0.050\times0.001\approx0.0001m/s,\\
        u_{r,A^2}&=\frac{u_{A^2}}{A^2}\times100\%\approx4\%
    \end{split}
    \]
    The results of $A^2$ are shown in Table \ref{adata2}.
    \begin{table}[!h] \small
        \centering
        \begin{tabular}{|c|c|c|c|}
            \hline
            No. & $A^2[m^2]$ & $u_{A^2}[m^2]$ & $u_{r,A^2}[\%]$\\ \hline
            1 & 0.0025 & 0.0001 & 4\\ \hline
            2 & 0.0100 & 0.0002 & 2\\ \hline
            3 & 0.0225 & 0.0003 & 1.3\\ \hline
            4 & 0.0400 & 0.0004 & 1.0\\ \hline
            5 & 0.0625 & 0.0005 & 0.8\\ \hline
            6 & 0.0900 & 0.0006 & 0.6\\ \hline
        \end{tabular}
        \caption{Results of $A^2$}\label{adata2}
    \end{table}

    Finally, by curve fitting we find the relation is that $v_{max}^2\propto A^2$. The slope is
    \begin{equation}\label{k/m}
        slope=22.22\pm0.31s^{-2}, \quad u_r=1.4\%.
    \end{equation}

    According to the formula of energy conservation in simple harmonic motion $\frac{1}{2}mv_{max}^2=\frac{1}{2}kA^2$, we obtain the theoretical value of the slope $k/m$ from the experimentally found $k$ (from Eq. \ref{keff}) and $m$ (from Table \ref{M}), which will be regarded as theoretical value. Here $m$ is the mass of the object with U-shape shutter and springs.
    \[
    \begin{split}
        k&=k_{eff}=4.47kg/s^2\\
        m&=m_{objU}=0.19392kg\\
        \frac{k}{m}&=\frac{4.47}{0.19392}=23.1s^{-2}.
    \end{split}
    \]
    Compared with the fitting slope in Eq. \ref{k/m}, we can calculate the deviation $\Delta slope$ and relative deviation $\Delta_r slope$
    \[
    \begin{split}
        \Delta slope&=22.2-23.1=-0.9s^2/kg,\\
        \Delta_r slope&=\frac{22.2-23.1}{23.1}\times 100\%=-4\%.
    \end{split}
    \]