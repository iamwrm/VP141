\section{Introduction}

The objective of this exercise is to study damped and driven oscillations in
mechanical systems using the Pohl resonator. For driven oscillations, we will
also observe and quantify the mechanical resonance phenomenon. 
    
If a periodically varying external force is applied to a damped harmonic
oscillator, the resulting motion is called forced (or driven) oscillations, and
the external force is called the driving force. Assuming that the driving force
is of the form 
    \[
        F=F_0(sin\omega t+\delta),
    \]
with the amplitude F 0 and angular frequency ω, the resulting steady-state
forced oscillations will be simple harmonic with the angular frequency equal to
that of the driving force. The amplitude of these steady-state oscillations
turns out to depend on the angular frequency of the the driving force, in
particular on how far it is from the natural angular frequency, and the damping
coefficient. The amplitude may become quite large, and this phenomenon is known
as the mechanical resonance. 

Another interesting property of driven steady-state oscillations is the fact
that there is a phase lag between the driving force and the displacement from
the equilibrium position of the oscillating particle. This phase lag reaches
$\pi/2$ (a quarter of the cycle) when the system is driven at the natural
angular frequency.
    
In this experiment, forced oscillation of a balance wheel will be studied. The
corresponding quantities (such as the force and the position) will be replaced
by their angular counterparts. 

The driving torque $\tau_{dr}=\tau_0cos\omega t$ and a damping torque
$\tau_f=-b\frac{d\theta}{dt}$, Also, we know the restoring torque
$\tau=-k\theta$, its equation od motion is of the form 

\begin{equation}
\label{1}
I\frac{d^2\theta}{dt^2}=-k\theta-b\frac{d\theta}{dt}+\tau_0cos\omega t,
\end{equation}

where I is the moment of inertia of the balance wheel, $\tau_0$ is the amplitude
of the driving torque, and $\omega$ is the angular frequency of the driving
torque. Introducing the symbols 

\[
\omega_0=\frac{k}{I},\quad 2\beta=\frac{b}{I}, \quad \mu=\frac{\tau_0}{I}, 
\]

Eq. \ref{1} can be rewitten as
\begin{equation}
\label{2}
\frac{d^2\theta}{dt^2}+2\beta\frac{d\theta}{dt}+\omega_0^2\theta
=\mu cos\omega t.
\end{equation}

Thw solution to Eq \ref{2} is
\[
\theta(t)=\theta_{tr}(t)+\theta_{st}cos(\omega t+\varphi),
\]
where the former term $\theta_{tr}$ denotes the transient solution that vanished
exponentially as $t\rightarrow \infty$. The steady-state  oscillation is with
the amplitude 
\[
\theta_{st}=\frac{\mu}{\sqrt{(\omega_0^2-\omega^2)^2+4\beta^2\omega^2}}
\]
For small values of the damping coefficient \beta, the resonance angular
frequency is close to the the natural angular frequency, and the amplitude of
steady-state oscillations becomes large. The dependence of both the amplitude
and the phase shift on the driving angular frequency are shown in the left and
right Figure 1, respectively, for different values of the damping coefficient.