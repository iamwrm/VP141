\section{Introduction}
Fluid viscosity is one of the most important properties of fluids, determining
the fluid’s flow. 
Motion of an object in a fluid is hindered by a drag force acting in the
direction opposite to the direction of motion, i.e. opposite to the object’s
velocity. 
The magnitude of the drag force is related to the shape and speed of the object
as well as to the internal friction in the fluid.
The method used in this lab is known as Stokes’ method and it a common and
simple method for characterizing transparent or translucent fluids with high
viscosity.


% theoretical
Motion of an object in a fluid is hindered by a drag force acting in the
direction opposite to the direction of motion, i.e. opposite to the object’s
velocity. 
The magnitude of the drag force is related to the shape and speed of the object
as well as to the internal friction in the fluid.
This internal friction can be quantified by a number known as the viscosity
coefficient $\mu$.
For a spherical object with radius R moving at speed v in an infinite volume of
a liquid, the magnitude of the drag force is usually modeled as linear in the
speed
$$  F_1 = 6 \pi \mu v R  $$

When a spherical object falls vertically downwards in a fluid, it is being acted
upon by the following three forces:
The viscous force \emph{$F_1$} and the buoyancy force \emph{$F_2$} both act
upwards, and the weight of the object \emph{$F_3$} is directed downwards.
The magnitude of the buoyancy force is
$$  F_2 = \frac{4}{3} \pi R^3 \rho_1 g $$
where $\rho_1$ is the density of the fluid and $g$ is the acceleration due to
gravity. The weight of the object
$$  F_3 = \frac{4}{3} \pi R^3 \rho_2 g $$




