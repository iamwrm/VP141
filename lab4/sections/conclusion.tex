\section{Conclusion and Discussion}
\subsection{The resonance method and the comparison method}
    In this experiment the spped of sound in the air was found by two means: the resonance method and the phase comparison method. The two results yielded the values
    \begin{equation}\label{res}
    \begin{split}
        v&=351.05\pm0.4 m/s,\quad u_{r,v}=0.10\%;\\
        v&=348.95\pm10 m/s,\quad u_{r,v}=3\%,
    \end{split}
    \end{equation}
    respectively. The resonance method obtains obviously smaller uncertainty. To compare these two experimentally found results, we need an objective standard.

    According to Bohn Dennis A. in the report "Environmental effects on the speed of sound", \emph{Journal of the Audio Engineering Society} P223-231, the speed of sound with respect to temperature holds the following equation
    \[
        c=331.45\sqrt{1+\frac{t}{273}}\quad\quad\quad\quad\cite{foo1},
    \]
    where $c$ is the sound of the speed and $t$ is the temperature in degrees Celsius.

    Since temperature we meausured was $\SI{24\pm1}{\degreeCelsius}$, the "theoretical" value of sound speed should be
    \[
        v=331.45\sqrt{1+\frac{24}{273}}=345.71m/s
    \]
    which is within the uncertainty interval of the results with the phase comparison method.

    Hence we can conclude that the results of the resonance method is \textbf{precise enough} but \textbf{not accurate enough}. 
    
    On the contrary, the results of the phase comparison method is \textbf{accurate enough} but \textbf{not precise enough}.\\

    The possible reason for errors in the resonance method is the system error. Since the uncertainty is quite small, there should exists a certain error between the experimental value and the theoretical value, probably becasue the frequency displayed on the screen had a certain deviation.

    The possible reason for errors in the phase comparison method is the reading error. We found that it was still changing in shape after we stop moving the calliper so that it's hard to tell whether it's the straight segment we want. Besides, the reading of calliper was inaccurate. 

\subsection{The time difference method for $v_{water}$}
    The experimentally found value is
    \[
        v_{water}=1515\pm20m/s,\quad u_{r,v}=1.5\%.
    \]

    According to the experiment results of N Bilaniuk and GSK Wong in the report "Speed of sound in pure water as a function of temperature", 
    \begin{table}[H]
        \centering
        \begin{tabular}{|c|c|c|c|c|c|}
        \hline
            $t(\SI{}{\degreeCelsius})$ & 23.8 & 23.9 & 24.0 & 24.1 & 24.2\\\hline
            $v(m/s)$ & 1493.440 & 1493.717 & 1493.992 & 1494.267 & 1494.541\\\hline
        \end{tabular}
        \caption{Speed of sound in water with respect to temperature\cite{foo2}}\label{water}
    \end{table}

    We find that our experimental value is relatively close to the standard value.

\subsection{Recommendations}
    For the resonance method, I think it'll be more accurate if there exists somthing like a prompting light to prompt the user when the voltage begins decreasing so that the user can stop the calliper in time and read the accurate value of length.