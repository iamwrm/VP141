\section{Conclusions and discussion}
    In this experiment the viscosity coefficient $\eta$ is measured by Stokes' method. We obtain the density of the metal balls by measuring its diameter and mass. We obtain the constant velocity from the traveling distance and time. We measure the inner diameter of the flask and the read the density of castor oil. Finally, the experimentally found $\eta$ of castor oil in the environment of $\SI{24}{\degreeCelsius}, 1 atm$ is
    \[
        \eta=0.679\pm 0.03\frac{kg}{m\cdot s}, \quad u_{r_{\eta}}=3.91\%.
    \]

    After the experiment, I find there are some operations probably leading to deviations in the fundamental Stokes' method. Firstly, I find it hard to confirm that the two beams are parallel. It'll be better if there exist a fixed equipment to measure the distance between the transmitting and the receiving end. Secondly, I find that among the directly measured data, the relative uncertainty of traveling time is the biggest. Stopwatch is indeed not accurate enough to eliminate the deviation. Therefore,the results can be more accurate if the time is recorded by computer when the ball blocks the beams.