\section{Introduction}
    The objective of the exercise is to measure the fluid viscosity, an important property of fluids, using Stoke's method.\\

    To analyze the free body diagram of a spherical object moving in a fluid, we find that the viscous force, the buoyancy force and the weight, where the first two forces act upwards and the last one  acts downwards.\\

    The magnitude of a drag force is related to the shape and speed of the objective as well as to the internal friction in the fluid. We use coefficient $\eta$ to quantify the internal friction in the fluid. Hence we build a model for the drag force (viscous force) in an infinite volume of a liquid.
    \[
        F_1=6\pi\eta vR
    \]
    The magnitude of the buoyancy force is
    \[
        F_2=\frac{4}{3}\pi R^3\rho_1g,
    \]
    where $\rho_1$ is the density of the fluid and g is the acceleration due to gravity. The weight of the object is
    \[
        F_3=\frac{4}{3}\pi R^3\rho_2g,    
    \]
    where $\rho_2$ is the density of the object. Since the three forces balance each other, then 
    \[
        F_1+F_2=F_3.
    \]
    Assuming that the object will be moving with constant speed $v_t$, we find from the equation that
    \[
        \eta=\frac{2}{9}gR^2\frac{\rho_2-\rho_1}{v_t}.
    \]
