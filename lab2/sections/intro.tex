\section{Introduction}
    The objective of the exercise is to measure the fluid viscosity, an important property of fluids, using Stokes' method.\\

    To analyze the free body diagram of a spherical object moving in a fluid, we find that the viscous force, the buoyancy force and the weight, where the first two forces act upwards and the last one  acts downwards.\\

    The magnitude of a drag force is related to the shape and speed of the objective as well as to the internal friction in the fluid. We use coefficient $\eta$ to quantify the internal friction in the fluid. Hence we build a model for the drag force (viscous force) in an infinite volume of a liquid.
    \begin{equation} \label{F1}
        F_1=6\pi\eta vR
    \end{equation}
    The magnitude of the buoyancy force is
    \[
        F_2=\frac{4}{3}\pi R^3\rho_1g,
    \]
    where $\rho_1$ is the density of the fluid and g is the acceleration due to gravity. The weight of the object is
    \[
        F_3=\frac{4}{3}\pi R^3\rho_2g,    
    \]
    where $\rho_2$ is the density of the object. Since the three forces balance each other, then 
    \begin{equation} \label{Balance}
        F_1+F_2=F_3.
    \end{equation}
    Assuming that the object is moving with constant speed $v_t$, we find from Eq. {\ref{Balance}} that
    \[
        \eta=\frac{2}{9}gR^2\frac{\rho_2-\rho_1}{v_t}.
    \]
    Considering that the velocity is constant, we substitute $\frac{s}{t}$ for $v_t$
    \begin{equation} \label{eta}
        \eta=\frac{2}{9}gR^2\frac{(\rho_2-\rho_1)t}{s},
    \end{equation}
    where $s$ is the distance travled in time $t$ with reaching the terminal speed.\\

    We also need to modify Eq.{\ref{F1}} since the volume of the fluid is not infinite. To eliminate the boundary effects due to the container. Assume that the radius of the infinitely long cylindrical container is $R_c$, then
    \[
        F_1=6\pi\eta vR(1+2.4\frac{R}{R_c})
    \]
    Eventually, the viscosity coefficient can be determined as
    \begin{equation} \label{eta2}
        \eta=\frac{2}{9}gR^2\frac{(\rho_2-\rho_1)t}{s}\frac{1}{1+2.4\frac{R}{R_c}}.
    \end{equation}
    
    Besides, the length $L$ may contribute to further corrections, which depends on thate ratio $R_c/L$.