\section{Measurement uncertainty analysis}
\subsection{Uncertainty of distance measurements}
    The uncertainty (of type-B) of a steel ruler used to measure the vertical distence between two beams is $\Delta_{S,B}=\Delta_{dev}=0.10\times10^{-3}m$. The distance is found by taking the average of 3 measurements. To estimate type-A uncertainty, the standard deviation of the average value is caluculated as
    \[
        s_{\bar{S}}=\sqrt{\frac{1}{n(n-1)}\sum_{i=1}^n(S_i-\bar{S})^2}.
    \]
    Uisng the data from Table \ref{distance} we find that $s_{\bar{S}}\approx 0.44\times10^{-3}m$. Considering $t_{0.95}=4.30$ for $n=3$, the type-A uncertainty is estimated as $\Delta_{S,A}=4.30\times0.441\times10^{-3}m\approx 1.89\times10^{-3}m$.\\
    Hence the combined uncertainty is
    \[
        u_{S}=\sqrt{\Delta_{S,A}^2+\Delta_{S,B}^2}=\sqrt{(1.89\times10^{-3})^2+(0.10\times10^{-3})^2}\approx 1.89\times10^{-3}m
    \]
    and the corresponding relative uncertainty is 
    \[
        u_{rS}=\frac{u_S}{\bar{S}}\times 100\%=1.05\%.
    \]
    The experimentally found $S$ is 
    \[
        S=180.2\pm 1.9 \times10^{-3}m,\quad u_{rS}=1.05\%.
    \]

\subsection{Uncertainty of time measurements}
    The uncertainty (of type-B) of a stopwatch used to measure the traveling time between two beams is $\Delta_{t,B}=\Delta_{dev}=0.01s$. The time is found by taking the average of 6 measurements. To estimate type-A uncertainty, the standard deviation of the average value is caluculated as
    \[
        s_{\bar{t}}=\sqrt{\frac{1}{n(n-1)}\sum_{i=1}^n(t_i-\bar{t})^2}.
    \]
    Uisng the data from Table \ref{time} we find that $s_{\bar{t}}\approx 0.04s$. Considering $t_{0.95}=2.57$ for $n=6$, the type-A uncertainty is estimated as $\Delta_{t,A}=2.57\times0.04s\approx 0.10s$.\\
    Hence the combined uncertainty is
    \[
        u_{t}=\sqrt{\Delta_{t,A}^2+\Delta_{t,B}^2}=\sqrt{(0.10)^2+(0.01)^2}\approx 0.10s
    \]
    and the corresponding relative uncertainty is 
    \[
        u_{rt}=\frac{u_t}{\bar{t}}\times 100\%=1.19\%.
    \]
    The experimentally found $t$ is 
    \[
        t=8.38\pm 0.10s,\quad u_{rt}=1.19\%.
    \]

\subsection{Uncertainty of measurements for the diameter of the balls}
    The uncertainty (of type-B) of a micrometer used to measure the diameter of the balls is $\Delta_{d,B}=\Delta_{dev}=0.004\ \times10^{-3}m$. The diameter is found by taking the average of 10 measurements. To estimate type-A uncertainty, the standard deviation of the average value is caluculated as
    \[
        s_{\bar{d}}=\sqrt{\frac{1}{n(n-1)}\sum_{i=1}^n(d_i-\bar{d})^2}.
    \]
    Uisng the data from Table \ref{balldiameter} we find that $s_{\bar{d}}\approx 0.0014\times10^{-3}m$. Considering $t_{0.95}=2.26$ for $n=10$, the type-A uncertainty is estimated as $\Delta_{d,A}=2.26\times0.0014\times10^{-3}\approx 0.0031\times10^{-3}m$.\\
    Hence the combined uncertainty is
    \[
        u_{d}=\sqrt{\Delta_{d,A}^2+\Delta_{d,B}^2}=\sqrt{(0.003\times10^{-3})^2+(0.004\times10^{-3})^2}\approx 0.005\times10^{-3}\ m
    \]
    and the corresponding relative uncertainty is 
    \[
        u_{rd}=\frac{u_d}{\bar{d}}\times 100\%=0.25\%.
    \]
    The experimentally found $d$ is 
    \[
        d=1.985\pm 0.005 \times10^{-3}m,\quad u_{rd}=0.25\%.
    \]

\subsection{Uncertainty of measurements for the inner diameter of the flask}
    The uncertainty (of type-B) of a calliper used to measure the inner diameter of the flask is $\Delta_{D,B}=\Delta_{dev}=0.02\times10^{-3}m$. The diameter is found by taking the average of 6 measurements. To estimate type-A uncertainty, the standard deviation of the average value is caluculated as
    \[
        s_{\bar{D}}=\sqrt{\frac{1}{n(n-1)}\sum_{i=1}^n(D_i-\bar{D})^2}.
    \]
    Uisng the data from Table \ref{innerdiameter} we find that $s_{\bar{D}}\approx 0.08\times10^{-3}m$. Considering $t_{0.95}=2.57$ for $n=6$, the type-A uncertainty is estimated as $\Delta_{D,A}=2.57\times0.08\times10^{-3}\approx 0.21\times10^{-3}m$.\\
    Hence the combined uncertainty is
    \[
        u_{D}=\sqrt{\Delta_{D,A}^2+\Delta_{D,B}^2}=\sqrt{(0.21\times10^{-3})^2+(0.02\times10^{-3})^2}\approx 0.21\times10^{-3}\ m
    \]
    and the corresponding relative uncertainty is 
    \[
        u_{rD}=\frac{u_D}{\bar{D}}\times 100\%=0.34\%.
    \]
    The experimentally found $D$ is 
    \[
        D=62.41\pm 0.21 \times10^{-3}m,\quad u_{rD}=0.34\%.
    \]

\subsection{Uncertainty of the density of the metal ball}
    Density of the metal balls can be found from
    \[
        \rho_2=\frac{m_0}{V}=\frac{m_0}{\frac{4}{3}\pi(\frac{d}{2})^3}=\frac{6m_0}{\pi d^3}\approx 8.285\times10^3 kg/m^3.
    \]
    In order to find the propagated uncertainty, first find the partial derivatives
    \[
        \frac{\partial\rho_2}{\partial m_0}=\frac{6}{\pi d^3},\quad
        \frac{\partial\rho_2}{\partial d}=\frac{18m_0}{\pi d^4}
    \]
    Hence, using Matlab we can obtain the propagated uncertainty by the formula
    \[
        u_{\rho_2}=\sqrt{(\frac{\partial\rho_2}{\partial m_0})^2(u_{m_0})^2+(\frac{\partial\rho_2}{\partial d})^2(u_d)^2}\approx 0.252\times10^3kg/m^3.
    \]
    Similarly, to find the relative uncertainty, first find the partial derivatives
    \[
        \frac{\partial\ln\rho_2}{\partial m_0}=\frac{1}{m_0},\quad
        \frac{\partial\ln\rho_2}{\partial d}=-\frac{3}{d}
    \]
    The relative uncertainty is
    \[
        u_{r\rho_2}=\frac{u_{\rho_2}}{\bar{\rho_2}}=\sqrt{(\frac{\partial \ln\rho_2}{\partial m_0})^2(u_{m_0})^2+(\frac{\partial \ln\rho_2}{\partial d})^2(u_d)^2}\approx 3.04\%
    \]
    Hence the experimentally $\rho_2$ is
    \[
        \rho_2=8.285\pm 0.252\times10^3kg/m^3, \quad u_{r\rho_2}=3.04\%.
    \]

\subsection{Uncertainty of the viscosity coefficient}
    The viscosity coeffient can be found from
    \[
        \eta=\frac{2}{9}gR^2\frac{(\rho_2-\rho_1)t}{s}\frac{1}{1+2.4\frac{R}{R_c}}\approx 0.6790\frac{kg}{m\cdot s}.
    \]
    In order to find the propagated uncertainty, first find the partial derivatives
    \[
    \begin{split}
        &\frac{\partial\eta}{\partial R}=- \frac{8R^{2}gt(\rho_2-\rho_1)}{15R_{c}S(\frac{2.4R}{R_{c}}+1)^2} + \frac{4Rgt(\rho_2-\rho_1)}{9S\frac{2.4R}{R_{c}}+1}\\
        &\frac{\partial\eta}{\partial \rho_2}=\frac{2R^{2} g t}{9S(\frac{2.4 R}{R_{c}} + 1)}\\
        &\frac{\partial\eta}{\partial \rho_1}=-\frac{2R^{2} g t}{9S(\frac{2.4 R}{R_{c}} + 1)}\\
        &\frac{\partial\eta}{\partial t}=\frac{2R^2 g (\rho_2-\rho_2)}{9S (\frac{2.4 R}{R_{c}}+1)}\\
        &\frac{\partial\eta}{\partial S}=-\frac{2R^2 g t (\rho_{2} - \rho_{1})}{S^{2} (\frac{2.4 R}{R_{c}} + 1)}\\
        &\frac{\partial\eta}{\partial R_c}=\frac{8R^{3} g t (\rho_{2} - \rho_{1})}{15R_{c}^{2} S (\frac{2.4 R}{R_{c}} + 1)^{2}}
    \end{split}
    \]
    Hence, using Matlab we can obtain the propagated uncertainty by the formula
    \[
    \begin{split}
        u_{\eta}&=\sqrt{(\frac{\partial\eta}{\partial R})^2(u_R)^2+(\frac{\partial\eta}{\partial \rho_2})^2(u_{\rho_2})^2+(\frac{\partial\eta}{\partial \rho_1})^2(u_{\rho_1})^2+(\frac{\partial\eta}{\partial u_t})^2(t)^2+(\frac{\partial\eta}{\partial S})^2(u_S)^2+(\frac{\partial\eta}{\partial u_{R_c}})^2(u_{R_c})^2}\\
        &\approx 0.02656\frac{kg}{m\cdot s}.
    \end{split}
    \]
    Similarly, to find the relative uncertainty, first find the partial derivatives
    \[
    \begin{split}
        &\frac{\partial\ln\eta}{\partial R}=\frac{9 s}{2 R^{2} g t \left(\rho_{2} - \rho_{1}\right)} \left(\frac{2.4 R}{R_{c}} + 1\right) \left(- \frac{8 R^{2} g t \left(\rho_{2} - \rho_{1}\right)}{15R_{c} s \left(\frac{2.4 R}{R_{c}} + 1\right)^{2}} + \frac{4 R g t \left(\rho_{2} - \rho_{1}\right)}{9s \left(\frac{2.4 R}{R_{c}} + 1\right)}\right)\\
        &\frac{\partial\ln\eta}{\partial \rho_2}=\frac{1}{\rho_2-\rho_1}\\
        &\frac{\partial\ln\eta}{\partial \rho_1}=-\frac{1}{\rho_2-\rho_1}\\
        &\frac{\partial\ln\eta}{\partial t}=\frac{1}{t}\\
        &\frac{\partial\ln\eta}{\partial s}=-\frac{1}{s}\\
        &\frac{\partial\ln\eta}{\partial R_c}=\frac{2.4 R}{R_{c}^{2} \left(\frac{2.4 R}{R_{c}} + 1\right)}
    \end{split}
    \]
    The relative uncertainty is
    \[
    \begin{split}
        u_{\eta}&=[(\frac{\partial\ln\eta}{\partial R})^2(u_R)^2+(\frac{\partial\ln\eta}{\partial \rho_2})^2(u_{\rho_2})^2+(\frac{\partial\ln\eta}{\partial \rho_1})^2(u_{\rho_1})^2\\
        &\quad+(\frac{\partial\ln\eta}{\partial u_t})^2(t)^2+(\frac{\partial\ln\eta}{\partial S})^2(u_S)^2+(\frac{\partial\ln\eta}{\partial u_{R_c}})^2(u_{R_c})^2]^\frac{1}{2}\\
        &\approx 3.912\%.
    \end{split}
    \]
    Hence the experimentally $\eta$ is
    \[
        \eta=0.6790\pm 0.02656\frac{kg}{m\cdot s}, \quad u_{r_{\eta}}=3.912\%.
    \]
