\section{Results}
\subsection{Measurements of the vertical distance}
    The average value of the vertical distance of two beams is calculated  based on the results presented in Table \ref{distance} as
    \[
        \bar{S}=\frac{1}{3}\sum_{i=1}^{3}S_i=(180.2\pm 1.9)\times10^{-3}m.
    \]
    \begin{table}[h]
        \centering
        \begin{tabular}{cc}
            \hline
            $Measurement$ & $Distance\ S\ [\times10^{-3}m] \pm\ 0.10\ [\times10^{-3}m]$\\
            \hline
            $S_1$ & 179.5\\
            $S_2$ & 180.0\\
            $S_3$ & 181.0\\     
            \hline
        \end{tabular}\\
        \caption{Distance measurement data}
        \label{distance}
    \end{table}

\subsection{Measurements of the traveling time}
    \begin{table}[htbp]
        \centering
        \begin{tabular}{cc}
            \hline
            $Measurement$ & $Time\ t\ [s] \pm\ 0.01\ [s]$\\
            \hline
            $t_1$ & 8.31\\
            $t_2$ & 8.35\\
            $t_3$ & 8.53\\
            $t_4$ & 8.41\\
            $t_5$ & 8.25\\
            $t_6$ & 8.44\\
            \hline
        \end{tabular}\\
        \caption{Time measurement data}
        \label{time}
    \end{table}
    The average value of the traveling time between two beams is calculated  based on the results presented in Table \ref{time} as
    \[
        \bar{t}=\frac{1}{6}\sum_{i=1}^{6}t_i=8.38\pm 0.10s
    \]

\subsection{Measurements for the diameters of the balls}
    \begin{table}[htbp]
        \centering
        \begin{tabular}{ccc}
            \hline
            $Measurement$ & $Initial\ readings[\times10^{-3}m]$ & $Diameters\ d[\times10^{-3}m] \pm 0.004[\times10^{-3}m]$\\
            \hline
            $d_1$ & -0.000 & 1.985\\
            $d_2$ & -0.000 & 1.990\\
            $d_3$ & -0.000 & 1.980\\
            $d_4$ & -0.000 & 1.990\\
            $d_5$ & -0.000 & 1.980\\
            $d_6$ & -0.000 & 1.985\\
            $d_7$ & -0.000 & 1.980\\
            $d_8$ & -0.000 & 1.980\\
            $d_9$ & -0.000 & 1.985\\
            $d_{10}$ & -0.000 & 1.990\\            
            \hline
        \end{tabular}\\
        \caption{Measurement data for diameters of the balls}
        \label{balldiameter}
    \end{table}
    The average value of the diameter of a ball is calculated  based on the results presented in Table \ref{balldiameter} as
    \[
        \bar{d}=\frac{1}{10}\sum_{i=1}^{10}d_i=(1.985\pm 0.005)\times10^{-3}m
    \]

\subsection{Measurements for the inner diameters of the flask}
    \begin{table}[htbp]
        \centering
        \begin{tabular}{ccc}
            \hline
            $Measurement$ & $Diameters\ D[\times10^{-3}m] \pm 0.02[\times10^{-3}m]$\\
            \hline
            $D_1$ & 62.46\\
            $D_2$ & 62.50\\
            $D_3$ & 62.32\\
            $D_4$ & 62.50\\
            $D_5$ & 62.36\\
            $D_6$ & 62.34\\           
            \hline
        \end{tabular}\\
        \caption{Measurement data for the inner diameter of the flask}
        \label{innerdiameter}
    \end{table}
    The average value of the inner diameter of the flask is calculated based on the results presented in Table \ref{innerdiameter} as
    \[
        \bar{D}=\frac{1}{10}\sum_{i=1}^{10}D_i=(62.41\pm0.21)\times10^{-3}\ m
    \]

\subsection{Measurements of other physical quanities}
    In single measurements, its combined deviation $u$ is equal to $\Delta_{dev}$.
    \begin{table}[htbp]
        \centering
        \begin{tabular}{ll}
            \hline\hline
            Density of the castor oil & $\rho_1=955\pm 1kg/m^3$\\
            \hline
            Mass of 40 metal balls & $m=1.357\pm 0.001\times10^{-3}kg$\\
            \hline      
            Temperature in the lab & $T=24\pm \SI{2}{\degreeCelsius}$\\
            \hline
            Acceleration due to gravity in the lab & $g=9.794m/s^2$\\        
            \hline\hline
        \end{tabular}\\
        \caption{Values of other physical quantities}
        \label{other}
    \end{table}

    Hence the mass of a single metal ball is $m_0=(0.03393\pm 0.001)\times10^{-3}kg$