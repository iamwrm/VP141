\section{Conclusions and Discussion}

In the experiment, the moment of inertia is found by measuring the angular acceleration of the turntable in different cases. 

By plotting and fitting the relation between $k\pi$ and $ t $, we found the $\beta$ for each situation of the instrument.
For we used MATLAB to do the fitting work, the uncertainty of the fitting curve can be easily seen as $\pm 5\%$.

By comparing the calculated value of the difference of $I_A$ and $I_B$, we can use the experiment to judge whether the parallel axis theorem holds.

The calculated value of $I$ is $md^2 =  165.8 g \times  (4.7560 cm - 6.0120cm )^2 + 165.8 g \times  (4.7610 cm - 6.0200cm )^2  = 0.0052436 kg\times m^2 $ and the value derived from the experiment is $ I_{A3B4} -I_{A1B2} =0.0055  kg\times m^2 -   0.0044  kg\times m^2 = 0.0011 kg\times m^2 $.
The relative uncertainty is $$ \frac{ 0.0055  kg\times m^2 - 0.0052436 kg \times m^2}{0.0055  kg\times m^2} = 4.66 \% $$

Thus, the parallel axis theorem holds very well.

The g