\section{Conclusion and Discussion}
    In this exercise, we measure the corresponding physics quantities of the moment of inertia and calculate it by experiment. After that, we calculate the theoretical values of $I$ according to the moment of inertia formula we derive by integral. By comparison, we try to verify the parallel axis (Steiner's) theorem.

    Here list our final experimental and theoretical data again.
    
    \begin{table}[H] \small
        \centering
        \begin{tabular}{|c|c|c|c|c|c|}
            \hline
            & Theoretical value [$\e{-3}kg\cdot m^2$] & Experimental value [$\e{-3}kg\cdot m^2$] & $u_I$ [$\e{-3}kg\cdot m^2$] & $u_{I,r}$ [$\%$]\\\hline
            $I_{disk}$ & 3.51 & 3.83 & 0.06 & 1.5\\\hline
            $I_{hoop}$ & 5.08 & 5.44 & 0.07 & 1.2\\\hline
            $I_{1,2}$ & 0.72 & 0.81 & 0.09 & 11\\\hline
            $I_{3,4}$ & 1.91 & 2.07 & 0.05 & 3\\\hline
        \end{tabular}
        \caption{Results for the moments of inertia}\label{data_i}
    \end{table}
    
    $For the disk$, the relative deviation is $(3.83-3.51)/3.51\times100\%=9\%$, which is out of its 0.95 confidence bound. However, we can basically know the moment of inertia through this experiment.

    $For the hoop$, the relative deviation is $$(5.44-5.08)/5.08\times100\%=7\%$$, which is also out of its 0.95 confidence bound, but still acceptable.

    $For the cylinder case$, we can verify the parallel axis theorem. In the first case, the relative deviation is $$(0.81-0.72)/0.72\times100\%=11\%$$, which is just on 0.95 confidence bound point. In fact it shows that in this case the angular acceleration obtains larger uncertainty. In the second case, the relative deviation is $$(2.07-1.91)/1.91\times100\%=8\%$$, which is out of the 0.95 confidence bound. The possible reasons include
    \begin{enumerate}
        \item Shaking and displacement of the cylinders.
        \item Displacement of rotating axis.
        \item Change of frictional torque.
    \end{enumerate}

    However, basically, the small deviation generally prove the parallel axis theorem.

    I find it interesting that all the experimental values are larger than the theoretical values. Based on the formula, the possible reason include the gravity and the frictional torque.
    
    The exact value of acceleration due to gravity might be smaller than the standard value I reference, which seems to strongly explain the phenomenon.

    On the other hand, during my experiment I find that the string we use is fairly thick compared to the uncertainty. When it's winded, the radius of cone pulley rally changes greatly, let alone its friction.

    For improvements, I suppose that we'd better obtain a more accurate value of $g$ and use a kind of fishing line, which is thin, strong and smooth, relatively. Also, we should eliminate the shaking of cylinders. Finally, we find that in the cylinder case, the moment of inertia is quite small. If the mass of cylinders are larger, the deviation will be smaller.
