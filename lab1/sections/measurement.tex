\section{Measurement Procedure}
\begin{enumerate}
    \item Measure the mass of the weight, the hoop, the disk and the cylinder, as well as the radius of the cone pulley and the cylinder (following the instructor's requirements). Calculate the moment of inertia of the hoop and the disk analytically.\\
    Use a reliable source to find the local value of the acceleration due to gravity in Shanghai.
    \item Turn the electronic timer on and switch it to mode 1-2 (single gate, multiple pulses).
    \item Place the instrument close to the edge of the desk and stretch the disk pulley arm outside so that the weight can move downwards unobstructed.
    \item Level the turntable with bubble level.
    \item Make the turntable rotating and press the start button on the timer. After at least 8 signals are recorded, stop the turntable and record the data in your data sheet.
    \item Attach the weight to one end of the string. Place the string on the disk pulley, thread through the hole in the arm, and wind the string around the third ring of the cone pulley. Adjust the arm holder so that the string goes through the center of the hole.
    \item Release the weight and the start the timer. Stop the turntable when the weight hits the floor. Write down the recorded data.
    \item The angular acceleration can be found by plotting $\theta=k\pi$ against $t$ and performing a quadratic fit using data processing software. (The magnitude of the angular acceleration is equal to the coefficient next to $t^2$ multiplied by two. The uncertainty of the angular acceleration can be read directly from the fitting result.)
\end{enumerate}