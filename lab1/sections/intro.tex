\section{Introduction}
    The objectives of this exercise include measuring the moment of inertia of a rigid body with the constant-torque method. The dependence of the moment of inertia on mass distribution and on the choice of the rotation axis will be studied. The parallel axis (Steiner's) theorem can be verified. This exercise also involves the operation of a photo-gate electronic timer.

    Moment of inertia of a rigid body about an axis characterizes the body's resistance (inertia) to change of angular velocity in rotation about that axis. It's determined by both its mass and mass distribution. If the rigid body obtains an irregular shape or non-uniformly distributed mass, we can easily use experimental methods to calculate it.

\subsection{Second law of dynamics for rotational motion}
    The second law of dynamics for rotational motion about a fixed axis is 
    \begin{equation}
        \tau_z=I\beta_z,
    \end{equation}
    relates the component of the torque $tau_z$ about the axis of rotation with the moment of inertia about this axis, and the angular acceleration component $\beta_c$. Therefore, the moment of inertia $I$ can be found with these two quantities measured.

    Also we know that the moment of inertia is additive. The moment of inertia of a combined rigid body $AB$ composed of $A$ and $B$, about the same axis of rotation, is
    \[
        I_{AB}=I_A+I_B.
    \]

\subsection{Parallel axis theorem}
    If the moment of inertia of a rigid body about the axis through the body's center of mass is $I_0$, the nfor any axis parallel to that, the moment of inertia is
    \begin{equation}
        I=I_0+md^2,
    \end{equation}
    where $d$ is the distance between the two axes. This theorem is known as Parallel axis theorem or Steiner's theorem.