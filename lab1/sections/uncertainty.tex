\section{Measurement Uncetainty Analysis}
\subsection{Uncertainty for distances}
    For the calliper measurements except that diameters of holes, we have four measurements for each quantity. To demonstrate the uncertainty, the \textbf{sample calculation} for diameters of the disk will be presented. (raw data see Table \ref{table_calliper})
    \[
    \begin{split}
        &\bar{\varnothing}=0.23992m\\
        &\overline{s_{\varnothing}}=\sqrt{\frac{1}{4(4-1)}\sum_{i=1}^4(\varnothing_i-\bar{\varnothing})^2}\\
        &=\sqrt{\frac{( 0.23994-0.23992)^2+(0.23990-0.23992)^2+(0.23992-0.23992)^2+(0.23992-0.23992)^2}{12}}\\
        &=8.1650\e{-6}m.\\
    \end{split}
    \]
    Then, we obtain the type-A uncertainty and type-B uncertainty respectively. Their square root of sum of squre contribute to the combined uncertainty.
    \[
    \begin{split}
        &t_{0.95}=3.18,\quad n=4,\\
        &\Delta_{A,\varnothing}=t_{0.95}\times\overline{s_{\varnothing}}\approx3\e{-5}m,\\
        &\Delta_{B,\varnothing}=2\e{-5}m,\\
        &u_{\varnothing}=\sqrt{\Delta_{A,\varnothing}^2+\Delta_{B,\varnothing}^2}=\sqrt{(3\e{-5})^2+(2\e{-5})^2}=3\e{-5}m,\\
        &u_{\varnothing,r}=\frac{u_{\varnothing}}{\varnothing}\times100\%=\frac{3\e{-5}}{0.23992}\times100\%=0.014\%
    \end{split}
    \]
    Hence, 
    \[
        \varnothing=0.23992\pm0.00003m,\quad u_{\varnothing,r}=0.014\%.
    \]
    Similarly, we are capable of calculating all the uncertainties (see Table \ref{data_diameter}).

    Then, for the uncertainty of distance from the rotating axis to the holes, we need to calcualte the propagated uncertainty. Here's the \textbf{sample calculation} for hole 1.
    \[
    \begin{split}
        &R_1=\frac{d_{1,in}+d_{1,out}}{2},\\
        &\frac{\partial R_1}{\partial d_{1,in}}=\frac{\partial R_1}{\partial d_{1,out}}=\frac{1}{2},\\
        &u_{d_{1,in}}=u_{d_{1,out}}=\Delta_{dev}=2\e{-5}m\\
        &u_{R_1}=\sqrt{(\frac{\partial R_1}{\partial d_{1,in}})^2(u_{d_{1,in}})^2+(\frac{\partial R_1}{\partial d_{1,out}})^2(u_{d_{1,out}})^2}\\
        &=\sqrt{\frac{1}{4}(2\e{-5})^2+\frac{1}{4}(2\e{-5})^2}=1.4\e{-5}m,\\
        &u_{R_1,r}=\frac{u_{R_1}}{R_1}\times100\%=\frac{1.4\e{-5}}{0.04505}=0.03\%
    \end{split}
    \]

    Hence, 
    \[
        R_1=0.04505\pm 0.00001m,\quad u_{R_1,r}=0.03\%.
    \]

    Similarly, we obtain all the uncertainties of distances between holes (see Table \ref{data_hole}).

\subsection{Uncertainty of mass measurements}
    \textbf{For example}, for the mass of the disk, its uncertainty is merely its type-B uncertainty.
    \[
    \begin{split}
        &u_{m_{disk}}=\Delta_{B}=1\e{-4}kg,\\
        &u_{m_{disk},r}=\frac{u_{m_{disk}}}{m_{disk}}\times100\%=0.02\%
    \end{split}
    \]
    All the uncertainties of mass between holes (see Table \ref{data_mass}).

\subsection{Uncertainty of angular acceleration}
    We will calculate the uncertainty of quadratic fitting by $u=t_{0.95}\cdot std/\sqrt{8-2}$, where $t_{0.95}=2.36$ for n=8. Hoever, the angular acceleration is twice of coefficient of quadratic item. \textbf{For example}, when $p_1=-0.03547\pm0.00099 rad/s^2$ ($p_1$ is the coefficient of quadratic item),
    \[
    \begin{split}
        &\beta=2p_1=2\times (-0.03547)=-0.071rad/s^2,\\
        &u_{\beta}=\sqrt{(\frac{\partial \beta}{\partial p_1})^2(u_{p_1})^2}=2u_{p_1}=0.002rad/s^2,\\
        &u_{\beta,r}=\frac{u_{\beta}}{\beta}\times100\%=\frac{0.002}{0.071}=3\%.
    \end{split}
    \]

    Similar calculation results are listed in Table \ref{data_1}, \ref{data_2}, \ref{data_3}, \ref{data_4}, \ref{data_5}.

\subsection{Uncertainty of the moment of inertia}
    According to Eq. \ref{equ_I}, we know that the uncertainty of the moment of inertia obtains propagated uncertainty.

    Here we have the \textbf{sample calcultion} for empty turntable.
    \[
    \begin{split}
        &\frac{\partial I_1}{\partial m}=\frac{R(g-R\beta_2)}{\beta_2-\beta_1},\\
        &\frac{\partial I_1}{\partial R}=\frac{mg-2mR\beta_2}{\beta_2-\beta_1},\\
        &\frac{\partial I_1}{\partial \beta_1}=\frac{mR(g-R\beta_2)}{(\beta_2-\beta_1)^2},\\
        &\frac{\partial I_1}{\partial \beta_2}=-\frac{mR(R\beta_1+g)}{(\beta_2-\beta_1)^2},\\
        \end{split}
    \]

    \[
    \begin{split}
        u_{I_1}&=\sqrt{(\frac{\partial I_1}{\partial m})^2(u_m)^2+(\frac{\partial I_1}{\partial R})^2(u_R)^2+(\frac{\partial I_1}{\partial \beta_1})^2(u_{\beta_1})^2+(\frac{\partial I_1}{\partial \beta_2})^2(u_{\beta_2})^2}\\
        &=\sqrt{(\frac{R(g-R\beta_2)}{\beta_2-\beta_1})^2(u_m)^2+(\frac{mg-2mR\beta_2}{\beta_2-\beta_1})^2(u_R)^2+(\frac{mR(g-R\beta_2)}{(\beta_2-\beta_1)^2})^2(u_{\beta_1})^2+(-\frac{mR(R\beta_1+g)}{(\beta_2-\beta_1)^2})^2(u_{\beta_2})^2}\\
        &=\sqrt{(0.130)^2(0.0001)^2+(0.280)^2(0.00003)^2+(0.004)^2(0.002)^2+(-0.004)^2(0.012)^2}\\
        &\approx5\e{-5}kg\cdot m^2.\\
        u_{I_1,r}&=\frac{u_{I_1}}{I_1}\times100\%=\frac{0.00005}{0.00707}=0.7\%.
    \end{split}
    \]

    Hence,
    \[
        I_1=(7.07\pm 0.05) \e{-3}kg\cdot m^2, \quad u_{I_1,r}=0.7\%.
    \]

    Simlarly, we can calculate uncertainties for the other combined moments of inertia (see Table \ref{data_I}).

    Finally, due to the additivity of the moment of inertia, we calculate the difference. We take the disk as the \textbf{example}.
    \[
    \begin{split}
        &I_{disk}=I_2-I_1,\\
        &\frac{\partial I_{disk}}{\partial I_2}=1,\\
        &\frac{\partial I_{disk}}{\partial I_1}=-1,\\
        &u_{I}=\sqrt{(\frac{\partial I_{disk}}{\partial I_2})^2(u_{I_2})^2+(\frac{\partial I_{disk}}{\partial I_1})^2(u_{I_1})^2}\\
        &=\sqrt{(3\e{-5})^2+(5\e{-5})^2}\approx0.06\e{-3}kg\cdot m^2.
    \end{split}
    \]

    Simlarly, we can calculate uncertainties for the other moments of inertia (see Table \ref{data_i}).