\documentclass[12pt,a4paper]{article}

\usepackage[letterpaper]{geometry}

\usepackage{times}
\geometry{top=1.0in, bottom=1.0in, left=1.0in, right=1.0in}

\usepackage{fancyhdr}
\pagestyle{fancy}
\lhead{}
\rhead{}
\lfoot{}
\rfoot{}

\renewcommand{\headrulewidth}{0pt} 
\renewcommand{\footrulewidth}{0pt} 

\setlength\headsep{0.333in}

\usepackage{fontspec}
\setmainfont{Times New Roman}

\usepackage{graphicx}

\usepackage{hyperref}

\usepackage{caption}

\usepackage{indentfirst}

\usepackage{setspace}



\begin{document}

\begin{titlepage}

\newcommand{\HRule}{\rule{\linewidth}{0.5mm}}

\center

\textsc{\LARGE UM - SJTU Joint Institute}\\[1cm]
\textsc{\Large Physics Laboratory I}\\[0.5cm]
\textsc{\large VP141}\\[0.5cm]

\HRule \\[0.4cm]
{
    \bfseries
    {\huge Exercise II}\\[0.3cm]
    {\large Measurement of Fluid Viscosity}\\[0.2cm]
    \HRule \\[1.5cm]
}

\begin{minipage}{0.6\textwidth}

\large
\emph{Name:}\\
Tianyi \textsc{Ge} \\

\emph{Student Number:}\\
516370910168 \\

\emph{Group:}\\
17\\

\emph{Instructor:}\\
Prof. Mateusz \textsc{Krzyzosiak}

\end{minipage}\\[3.5cm]

{\large \today}\\[2cm]

\vfill

\end{titlepage}

\newpage

\section{Theoretical Background}

Moment of inertia of a rigid body about an axis is a quantitative
characteristics that defines the body’s resistance (inertia) to a change of
angular velocity in rotation about that axis. 
This characteristics of the rigid body rotating about a fixed axis is determined
not only by the mass of the body, but also by its distribution. 
The moment of inertia of a rigid body about a certain rotation axis can be
calculated analytically. 
However, if the body has irregular shape or non-uniformly distributed mass, the
calculation may be di cult.
Experimental methods turn out to be more useful in such cases.

\section{Apparatus}

\section{Procedure}

\section{Calculations and Results}

\section{Measurement Uncertianty Analysis}

\section{Conclusions and Discussion}

% data sheet

\end{document}