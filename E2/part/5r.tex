\section{Result}
\subsection{Distance Between Two Lasar Device}

\begin{table}[H]
  \centering
  \begin{tabular}{|l|c |c |c |}
    \hline
    \multicolumn{4}{|c|}{distance $S$ [mm] $\pm$ 1 [mm]} \\
    \hline
    S &  Ruler Start Point & Ruler End Point & Calculated Length \\
    \hline
    $S_1$ & 30.0 & 177.0 & 147.0 \\ \hline
    $S_2$ & 60.0 & 207.0 & 147.0 \\ \hline
    $S_3$ & 40.0 & 186.5 & 146.5 \\ \hline
  \end{tabular}
  \caption{Distance measurement data.}
\end{table}

Then we can find
$$  \bar{S} = \frac{1}{3} \sum_{k=1}^{3} S_k = 146.8 mm \pm 1 mm   $$
$$  u_{S,r} = 0.68 \%  $$ 


\subsection{Time Measurement}

\begin{table}[H]
  \centering
  \begin{tabular}{|l|c|}
    \hline
    \multicolumn{2}{|c|}{time $t$ [s] $\pm$ 0.01 [s] } \\
    \hline
    $t_1$ & 6.75 \\ \hline
    $t_2$ & 6.82 \\ \hline
    $t_3$ & 6.91 \\ \hline
    $t_4$ & 6.88 \\ \hline
    $t_5$ & 6.91 \\ \hline
    $t_6$ & 6.84 \\ \hline
  \end{tabular}
  \caption{Time measurement data.}
\end{table}


Then we can find
$$  \bar{t} = \frac{1}{6} \sum_{k=1}^{6} t_k =6.85 s \pm 0.01 s   $$
$$  u_{t,r} = 0.15 \%  $$ 

\subsection{The Diameters of The Balls}

The initial reading of the meter is 0.38 mm.
Thus, the raw data of measurement should firstly minus 0.38 mm, and is presented
as following, 

\begin{table}[H]
  \centering
  \begin{tabular}{|p{2cm}|p{3cm}||p{2cm} |p{3cm} |}
    \hline
    \multicolumn{4}{|c|}{diameter $d$ [mm] $\pm$ 0.005 [mm]  } \\
    \hline
    $d_1$ & 1.995 & $d_6$ & 1.995 \\ \hline
    $d_2$ & 1.995 & $d_7$ & 2.000 \\ \hline
    $d_3$ & 2.000 & $d_8$ & 1.800 \\ \hline
    $d_4$ & 1.995 & $d_9$ & 1.995 \\ \hline
    $d_5$ & 2.000 & $d_{10}$ & 1.995 \\ \hline
  \end{tabular}
  \caption{Ball diameter measurement data.}
\end{table}

Then we can find
$$  \bar{d} = \frac{1}{10} \sum_{k=1}^{10} d_k = 1.977  mm \pm  0.005 mm   $$
$$  u_{t,r} =  0.25 \%  $$ 

\subsection{The Inner Diameter of The Flask}

\begin{table}[H]
  \centering
  \begin{tabular}{|p{1cm}|c|}
    \hline
    \multicolumn{2}{|c|}{diameter $D$ [mm] $\pm$ 0.02 [mm]  } \\
    \hline
    $D_1$ & 61.40 \\ \hline
    $D_2$ & 61.46 \\ \hline
    $D_3$ & 61.20 \\ \hline
    $D_4$ & 61.36 \\ \hline
    $D_5$ & 61.20 \\ \hline
    $D_6$ & 61.50 \\ \hline
  \end{tabular}
  \caption{Flask diameter measurement data.}
\end{table}


Then we can find
$$  \bar{D} = \frac{1}{6} \sum_{k=1}^{6} D_k =  61.3533 mm \pm  0.02 mm   $$
$$  u_{D,r} =   0.03\%  $$ 


\subsection{Other Physical Quantities}

\begin{table}[H]
  \centering
  \begin{tabular}{|c|c|}
    \hline
    density of the castor oil $ \rho_1 [g/cm^3] \pm 0.001 [g/cm^3] $ & 0.955  \\ \hline 
    mass of 40 metal balls $ m  [g] \pm 0.001 [g] $ & 1.357 \\ \hline
    temperature in the lab $ T  [\circ C] \pm 2 [\circ C] $ & 25 \\ \hline
    acceleration due to gravity in the lab $ g [m/s^2] $ & 9.794 \\ \hline
  \end{tabular}
  \caption{Other Physical Quantities Measurement}
\end{table}

\subsection{Calculation of Density of One Ball}

The mass of one  metal ball can be calculated as
$$  m_0 = \frac{m}{40} = \frac{1.357 \times 10^{-3} }{40} = 3.3925 \times 10^-5 kg
\pm (2.500 \times 10^-8) kg $$ 
We can furtherly get the density,
$$ \rho_2 = \frac{m_0}{\frac{1}{6} \pi d^3} = \frac{3.3925 \times 10^-5
}{\frac{1}{6} \times 3.151593 \times (1.977 \times 10^-3)^3 } = 8.385 \times
10^3 kg/m^3 \pm (5.756 \times 10  ) kg/m^3  $$
$$  u_{\rho_2,r} =   0.6875\%  $$ 

\subsection{Calculation For The Viscosity Coefficient}
From the last equation in introduction part
\begin{multline*}
\mu = \frac{2}{9} g R^2 \frac{( \rho_2 - \rho_1 ) t  }{s} (1 + 2.4
\frac{R}{R_c})  =  \frac{2 \times (1.977 \times 10^-3 \times \frac{1}{2})^2
  (8.385\times 10^3 - 0.595 \times 10^3 ) \times 9.974 \times 6.85 }{9 \times
  146.8\times 10^-3 \times (1 + 2.4 \times \frac{1.977}{61.3533})} \\
 = 0.7307 Pa \times s \pm (6.8329 \times 10^-3 ) Pa \times s 
 \end{multline*}
$$  u_{\mu,r} =  0.93512 \%  $$ 

