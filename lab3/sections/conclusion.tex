\section{Conclusion}
    In this experiment, we study the simple harmonic motion.
    
    Firstly, using measurements of \emph{Jolly balance} we get the spring constants:
    \[
    \begin{split}
        k_1&= 2.22\pm 0.02 kg/s^2,\quad u_{r,k_1}=0.9\%,\\
        k_2&= 2.25\pm 0.01 kg/s^2,\quad u_{r,k_2}=0.4\%,\\
        k_3&= 1.11\pm 0.017kg/s^2,\quad u_{r,k_3}=1.5\%.
    \end{split}
    \]
    We discuss the theoretical value of $k_3$ and find that it's within the confidence interval, which means the experiment process is precise.\\

    Secondly, we study the relation between $T^2$ and $M$ and find that $T^2\propto M$ and it does not depend on whether it's horizontal or not. Then we compare the slope of $T^2\ vs.\ M$ with "theoretical value". It's also within the confidence interval, confirming the experiment precision.
    \[
    \begin{split}
        slope_e&=8.74\pm 0.08s^/kg\\
        slope_t&=8.82s^2/kg\\
    \end{split}
    \]

    After that, we discuss the relation between $T$ and $A$. The correlation coefficient is $-0.39$. Based on the interpretation of correlation coefficient(see Table \ref{rule}), we find that $-0.39$ is within the range of "low correlation". I suppose that it's because the period of one of the points in Figure \ref{at} is too small. Thus the correlation coefficient is negative. Considering experimental procedure, I think it's just random error and can be eliminated if we have more times of experiments.
    \begin{table}
        \centering
        \begin{tabular}{|c|c|}
            \hline \hline
            \textbf{Absolute value of correlation} & \textbf{Interpretation} \\\hline
            (0.90, 1.00) & very high correlation\\\hline
            (0.70, 0.90) & high correlation\\\hline
            (0.50, 0.70) & moderate correlation\\\hline
            (0.30, 0.50) & low correlation\\\hline
            (0.00, 0.30) & negligible correlation\\\hline\hline
        \end{tabular}
        \caption{Rule of interpretation of correlation coefficient \cite{foo1}}\label{rule}
    \end{table}\\

    Finally, we discuss the relation between $v_{max}$ and $A$. It's found that $v_{max}^2\propto A^2$. We compare the experimental slope $slope_e$ and the "theoretical slope" $slope_t$.
    \[
    \begin{split}
        slope_e&=22.22\pm0.31s^{-2}\\
        slope_t&=23.1s^{-2}.
    \end{split}
    \]
    It shows that the theoretical slope is out of the confidence interval. The experimental value is too small. It reminds me that during the experiment for each time I released the object, the distance it reached after a period was always less than the initial amplitude, which causes the kinetic energy is smaller than we expect. The possible reason might be the frictional force or the spring constants. The fiction between the springs and the hooks on the track ends might affect that, too. Besides, at the beginning the object needs more force to be at rest than that during the oscillation.
    
    I suggest that the U-shape shutter can be designed to be more narrow so that $v_{max}$ can be more precise; the program in the timer should be optimized to automatically record the proper average time in which the shutter travels, for example, recording the time intervals if the gaps between three consecutive readings are smaller than $0.5ms$.