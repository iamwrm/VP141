\section{Measurement Procedure}
\subsection{Spring constant}
    \begin{enumerate}
        \item Adjust the Jolly balance to vertical and attach the spring Add a 20g preload and adjust knob $I_1$ and $I_2$ to make sure the mirror can move freely through the tube. Check if the balance parallel to the spring.
        \item Adjust knob $G$ to make the three lines in the tube coincide.
        \item Reading the reading on the scale, add mass from 1 to 6 and record $L_i$ in order.
        \item Estimate the spring constant $k_1$ by fitting.
        \item Repeat the measurements with spring 2. Calculate $k_2$.
        \item Remove the preload and repeat the measurements for spring 1 and 2 connected. Calculate $k_3$.
    \end{enumerate}
\subsection{Relation between oscillation period $T$ and the mass of the oscillator $M$}
    \begin{enumerate}[(a)]
        \item Adjust the air track so that it's horizontal.
            \begin{enumerate}[1.]
                \item Turn on the air track and check whether there's blocked holes.
                \item Place the cart on the track. Adjust the track with the knob on the side with a single one.
            \end{enumerate}
        \item Horizontal air track
            \begin{enumerate}[1.]
                \item Attach springs to the sides of the cart and set up the I-shape shutter. Make sure that the photoelectric gate is at the equilibrium position.
                \item Set the timer into "T" mode. Add weight in order and release it with a caliper. Record the correspoding time intervals for 10 periods.
                \item Analyze the relation between $M$ and $T$ by plotting a graph.
            \end{enumerate}
        \item Inclined air track
            \begin{enumerate}[1.]
                \item Add three plastic plates under the air track everytime. Repeat the steps in (b).
                \item Analyze the relation between $M$ and $T$ by plotting a graph.
            \end{enumerate}
    \end{enumerate}
\subsection{Relation between the oscillation period and the amplitude}
    \begin{enumerate}
        \item Keep the mass of the cart unchanged and change the amplitude (choose 6 different values). The amplitude is about 5.0/ 10.0/ 15.0/ ... /30.0 cm.
        \item Apply linear fit to the data and comment on the relation between the oscillation period $T$ and the amplitude $A$ based on the correlation coefficient $\gamma$.
    \end{enumerate}
\subsection{Relation between the maximum speed and the amplitude}
    \begin{enumerate}
        \item Keep the mass of the cart unchanged and change the amplitude from $5.0\ to\ 30.0 cm$ with gap of $5.0cm$.
        \item Measure $v_{max}$ for different amplitudes.
    \end{enumerate}

\subsection{Mass measurement}
    \begin{enumerate}
        \item Adjust the electronic balance every time before you use it. The level bubble should be in the center of the circle.
        \item Add weights according to a fixed order. Weigh the cart with the I-shape shutter and with the U-shape shutter. Measure the mass of spring 1 and spring 2.
        \item Record the data only after the circular symbol on the scales display disappears.
    \end{enumerate}

\subsection{Devices Precision}
    The precision of the devices are shown in Table \ref{precision}.
    \begin{table}
        \centering
        \begin{tabular}{|l|c|c|}
            \hline
            Devices & Precision & Unit\\ \hline
            Jolly balance & 0.01 & [cm]\\ \hline
            Ruler & 0.1 & [cm]\\ \hline
            Timer for periods & 0.1 & [ms]\\ \hline
            Timer for maximum speed & 0.01 & [ms]\\ \hline
            Calliper & 0.02 & [mm]\\ \hline
            Electronic scale & 0.01 & [g]\\ \hline
        \end{tabular}
        \caption{Devices precision}\label{precision}
    \end{table}