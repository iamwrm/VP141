\section{Measurement Uncertainty Analysis}
\subsection{Uncertainty of spring constants}
    To determine the uncertainty of $\Delta_L=L_i-L_0$, we need that
    \[
    \begin{split}
        \frac{\partial \Delta_L}{\partial L_i}&=1,\\
        u_{L_i}&=\Delta_{dev}=1\times10^{-4}m,\\
        u_{\Delta_L}&=\sqrt{(\frac{\partial \Delta_L}{\partial L_i})^2\cdot (u_{L_i})^2}=\sqrt{(1)^2\times (1\times10^{-4})^2}=1\times10^{-4}m.\\
    \end{split}
    \].
    Similarly, $u_{W}$ of $W=mg$ is
    \[
    \begin{split}
        \frac{\partial W}{\partial m}&=g=9.794kg\cdot m/s^2,\\
        u_m&=\Delta_{dev}=1\times10^{-5}kg,\\
        u_{W}&=\sqrt{(\frac{\partial W}{\partial m})^2\cdot (u_m)^2}=\sqrt{(9.794)^2\times (1\times10^{-5})^2}\approx 0.1\times10^{-3}N.\\
    \end{split}
    \].
    The error of $\Delta_L$ and $W$ will be shown in errorbar.
\subsubsection{Uncertainty of fitting spring constants}
    The standard deviation of least-squares method is calculated by
    \[
        \sigma_W=\sqrt{\frac{1}{k-n}\sum_{i=1}^{k}\varepsilon_i^2}
    \],
    where $k$ is the number of measurements and $n$ is the number of unknown quantity. In this experiment, we need to figure out both $k$ and $b$ so $n$ should be 2 here. Considering spring 1, we plug in the data and obtain
    \[
    \begin{split}
        \sigma_W&\approx \sqrt{\frac{1}{6-2}\times [(0.288)^2+(-0.360)^2+(-0.285)^2+(0.0290)^2+(0.765)^2+(-0.437)^2]\times 10^{-6}}\\
        &=5.17\times10^{-4} N
    \end{split}
    \]
    Therefore, the standard deviation of the slop estimate
    \[
    \begin{split}
        \overline{\Delta_L^2}&=\frac{1}{6}[(0.0215)^2+(0.0431)^2+(0.0645)^2+(0.0853)^2+(0.1060)^2+(0.1278)^2]\\[0.2cm]
        &\approx 0.00689 m^2,\\[0.4cm]
        \overline{\Delta_L}^2&=[\frac{1}{6}[0.0215+0.0431+0.0645+0.0853+0.1060+0.1278]^2\\[0.2cm]
        &\approx 0.00558 m^2,\\[0.4cm]
        \sigma_{k_1}&=\frac{\sigma_W}{\sqrt{\overline{\Delta_L^2}-\overline{\Delta_L}^2}}=\frac{5.17\times10^{-4}}{\sqrt{0.00689-0.00558}}\\[0.2cm]
        &\approx14.3\times10^{-3} kg/s^2.
    \end{split}
    \]
    %and the standard deviation of the intercept estimate
    %\[
    %   \sigma_{b_1}=\sqrt{\overline{\Delta_L^2}}\sigma_{k_1}=\sqrt{0.00689}\times14.3\times10^{-3}\approx1.19\times10^{-3}N
    %\]
    Thus, the 0.95-confidence deviation is 
    \[
    \begin{split}
        u_{k_1}&=\frac{t_{0.95}}{\sqrt{n-2}}\sigma_{k_1}=\frac{2.57}{\sqrt{6-2}}\times14.3\times10^{-3}= 0.0184\approx 0.02kg/s^2.\\[0.4cm]
        %u_{b_1}&=\frac{t_0.95}{\sqrt{n}}\sigma_{b_1}=\frac{2.57}{\sqrt{6}}\times1.19\times10^{-3}\approx12.5\times10^{-4}N.
    \end{split}
    \]
    
    Similarly, we obtain $u_{k_2}$ and $u_{k_3}$.
    \[
    \begin{split}
        \sigma_{k_2}&=0.008kg/s^2,\quad
        u_{k_2}=\frac{t_{0.95}}{\sqrt{n-2}}\sigma_{k_2}=0.01kg/s^2.\\
        %\sigma_{b_2}&=0.006N,\quad\quad\quad
        %u_{b_2}=\frac{t_{0.95}}{\sqrt{n}}\sigma_{b_2}=0.007N.\\
        \sigma_{k_3}&=0.013kg/s^2,\quad
        u_{k_3}=\frac{t_{0.95}}{\sqrt{n-2}}\sigma_{k_3}=0.017kg/s^2.\\
        %\sigma_{b_3}&=0.002N,\quad\quad\quad
        %u_{b_3}=\frac{t_{0.95}}{\sqrt{n}}\sigma_{b_3}=0.002N.\\
    \end{split}
    \]
    
    The relative uncertainty can be calculated by $u_r=u/\bar{X}\times100\%$.
    For example,
    \[
        u_{r,k_1}=\frac{0.02}{2.22}\times100\%=0.9\%.
    \]
    Finally, the experimentally found $k_1$, $k_2$ and $k_3$ is
    \[
    \begin{split}
        k_1&= 2.22\pm 0.02 kg/s^2,\quad u_{r,k_1}=0.9\%\\
        k_2&= 2.25\pm 0.01 kg/s^2,\quad u_{r,k_2}=0.4\%\\
        k_3&= 1.11\pm 0.017kg/s^2,\quad u_{r,k_3}=1.5\%
    \end{split}
    \]

\subsubsection{Uncertainty of spring series' constants}
    To determine the theoretical value of $k_3$, we have the equations
    \[
    \begin{split}
        F&=k_1\Delta_{L_1},\\
        F&=k_2\Delta_{L_2},\\
        F&=k_3^{'}(\Delta_{L_1}+\Delta_{L_2}),
    \end{split}
    \]
    whose solution is $k_3^{'}=\frac{k_1k_2}{k_1+k_2}$. The theoretical value of $k_3$ is
    \[
        k_3^{'}=\frac{k_1k_2}{k_1+k_2}=\frac{2.22\times2.25}{2.22+2.25}=1.12kg/s^2.
    \]
    The propagated uncertainty of $k_3$ is estimated by the formula
    \[
    \begin{split}
        u_{k_3^{'}}&=\sqrt{(\frac{\partial k_3^{'}}{\partial k_1})^2(u_{k_1})^2+(\frac{\partial k_3^{'}}{\partial k_2})^2(u_{k_2})^2}\\[0.3cm]
        &=\sqrt{((\frac{k_2}{k_1+k_2})^2)^2(u_{k_1})^2+((\frac{k_1}{k_1+k_2})^2)^2(u_{k_2})^2}\\[0.3cm]
        &=\sqrt{(\frac{2.25}{2.22+2.25})^4(0.02)^2+(\frac{2.22}{2.22+2.25})^4(0.01)^2}\\[0.3cm]
        &=0.006kg/s^2
    \end{split}
    \]
    and the relative uncertainty is
    \[
        u_{rk_3^{'}}=\frac{u_{k_3^{'}}}{\overline{k_3^{'}}}=\frac{0.006}{1.12}=0.5\%.
    \]

    Hence the theoretical value of $k_3^{'}$ is
    \[
        k_3^{'}=1.12+0.006 kg/s^2,\quad u_{rk_3^{'}}=0.5\%.
    \]
    
    Compared with $k_3$ we obtain from curve fitting, we can calculate the deviation $\Delta_{k_3}$ between $k_3$ and $k_3^{'}$ and the relative deviation $\Delta_{r,k_3}$.
    \[ 
    \begin{split}
        \Delta_{k_3}&=k_3-k_3^{'}=1.11-1.12=-0.01kg/s^2,\\
        \Delta_{r,k_3}&=\frac{k_3-k_3^{'}}{k_3^{'}}=\frac{1.11-1.12}{1.12}=-0.9\%.
    \end{split}
    \]

    Recall that the 0.95-confidence interval of $k_3$ is $(1.093,1.127)$. The theoretical value $k_3^{'}=1.12+0.006 kg/s^2$ is indeed in this interval, which means the uncertainty analysis of $k_3$ is reasonable.\\
    
\subsection{Uncertainty of }
    From the formula of oscillation period $T=2\pi\sqrt{\frac{M}{k}}$ we know that the slope of $T^2\ vs.\ M$ is
    \begin{equation}\label{slope}
        slope=\frac{T^2}{M}=\frac{4\pi^2}{k},
    \end{equation}
    where k is the effective spring constant. In this experiment, the effective spring constant can be calculated by
    \[ 
    \begin{split}
        F&=k_1\Delta x+k_2\Delta x,\\
        F&=k_{eff}\Delta x.
    \end{split}
    \]
    Thus we get $k_{eff}=k_1+k_2$. We take the reults of $k_1$ and $k_2$ as the theoretical value, then
    \[
        k_{eff}=k_1+k_2=2.22+2.25=4.47kg/s^2.
    \]
    Plugging it into Eq. \ref{slope}, then we get the theoretical slope $slope_t=4\times 3.14^2/4.47=8.82s^2/kg$
    By curve fitting, we get the slopes
    \[
    \begin{split}
        slope_{hor}&=8.72\pm 0.05s^2/kg,\\
        slope_{inc1}&=8.73\pm 0.14s^2/kg,\\
        slope_{inc2}&=8.76\pm 0.17s^2/kg,
    \end{split}
    \]
    where the uncertainty is calcualted by $t_0.95/\sqrt{6-2}\times \sigma$.
    Hence the experimental value of the slope is $slope_e=\overline{slope}=8.74s^/kg$.
    Similarly, we get the propagated and relative uncertainty of slope, which is
    \[
    \begin{split}
        u_{slope}&=\sqrt{(\frac{\partial slope}{\partial slope_{hor}})^2\cdot(u_{slope_{hor}})^2+(\frac{\partial slope}{\partial slope_{inc1}})^2\cdot(u_{slope_{inc1}})^2+(\frac{\partial slope}{\partial slope_{inc2}})^2\cdot(u_{slope_{inc2}})^2}\\
        &=\sqrt{(\frac{1}{3})^2\cdot(0.05)^2+(\frac{1}{3})^2\cdot(0.14)^2+(\frac{1}{3})^2\cdot(0.17)^2}=0.08s^2/kg.\\
        u_{r,slope}&=\frac{u_{slope}}{\overline{slope}}\times100\%=\frac{0.08}{8.74}\times100\%=0.9\%.
    \end{split}
    \]

    Hence, $slope_e=\overline{slope}=8.74\pm 0.08s^/kg, \quad u_{r,slope}=0.9\%$. Compared with the "theoretical" value of $slope_t=8.82s^2/kg$, we calculate the deviation $\Delta_slope$ and relative deviation $\Delta_{r,slope}$
    \[
    \begin{split}
        \Delta_slope&=8.74-8.82=-0.08s^2/kg,\\
        \Delta_{r,slope}&=\frac{8.74-8.82}{8.82}\times 100\%=-0.9\%.
    \end{split}
    \]
    The theoretical value is just on the boundary of uncertainty interval. It shows that our experimental values are relatively reliable.

