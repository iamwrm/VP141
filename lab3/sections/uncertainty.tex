\section{Measurement Uncertainty Analysis}
\subsection{Uncertainty of spring constants}
    To determine the uncertainty of $\Delta L=L_i-L_0$, we need that
    \[
    \begin{split}
        &\frac{\partial \Delta L}{\partial L_i}=1,\quad \frac{\partial \Delta L}{\partial L_0}=-1,\\
        &u_{L_i}=u_{L_0}=\Delta_{dev}=1\times10^{-4}m,\\
        &u_{\Delta L}=\sqrt{(\frac{\partial \Delta L}{\partial L_i})^2\cdot (u_{L_i})^2+(\frac{\partial \Delta L}{\partial L_0})^2\cdot (u_{L_0})^2}=\sqrt{2}\times 1\times10^{-4}\approx1.4\times10^{-4}m.\\
    \end{split}
    \].
    Similarly, $u_{W}$ of $W=mg$ is
    \[
    \begin{split}
        \frac{\partial W}{\partial m}&=g=9.794kg\cdot m/s^2,\\
        u_m&=\Delta_{dev}=1\times10^{-5}kg,\\
        u_{W}&=\sqrt{(\frac{\partial W}{\partial m})^2\cdot (u_m)^2}=\sqrt{(9.794)^2\times (1\times10^{-5})^2}\approx 0.1\times10^{-3}N.\\
    \end{split}
    \].
    The errors of $\Delta L$ and $W$ have been shown in Figure \ref{k_1}, \ref{k_2} and \ref{k_3}.
\subsubsection{Uncertainty of fitting spring constants}
    The standard deviation of least-squares method is calculated by
    \[
        \sigma_W=\sqrt{\frac{1}{k-n}\sum_{i=1}^{k}\varepsilon_i^2},
    \]
    where $k$ is the number of measurements and $n$ is the number of unknown quantity. In this experiment, we need to figure out both $k$ and $b$ so $n$ should be 2 here. Here's the \textbf{sample calculation}. Considering spring 1, we plug in the data and obtain
    \[
    \begin{split}
        \sigma_W&\approx \sqrt{\frac{1}{6-2}\times [(0.288)^2+(-0.360)^2+(-0.285)^2+(0.0290)^2+(0.765)^2+(-0.437)^2]\times 10^{-6}}\\
        &=5.17\times10^{-4} N
    \end{split}
    \]
    Therefore, the standard deviation of the slop estimate
    \[
    \begin{split}
        \overline{\Delta L^2}&=\frac{1}{6}[(0.0215)^2+(0.0431)^2+(0.0645)^2+(0.0853)^2+(0.1060)^2+(0.1278)^2]\\[0.2cm]
        &\approx 0.00689 m^2,\\[0.4cm]
        \overline{\Delta L}^2&=[\frac{1}{6}[0.0215+0.0431+0.0645+0.0853+0.1060+0.1278]^2\\[0.2cm]
        &\approx 0.00558 m^2,\\[0.4cm]
        \sigma_{k_1}&=\frac{\sigma_W}{\sqrt{\overline{\Delta L^2}-\overline{\Delta L}^2}}=\frac{5.17\times10^{-4}}{\sqrt{0.00689-0.00558}}\\[0.2cm]
        &\approx14.3\times10^{-3} kg/s^2.
    \end{split}
    \]
    %and the standard deviation of the intercept estimate
    %\[
    %   \sigma_{b_1}=\sqrt{\overline{\Delta L^2}}\sigma_{k_1}=\sqrt{0.00689}\times14.3\times10^{-3}\approx1.19\times10^{-3}N
    %\]
    Thus, the 0.95-confidence deviation is 
    \[
    \begin{split}
        u_{k_1}&=\frac{t_{0.95}}{\sqrt{n-2}}\sigma_{k_1}=\frac{2.57}{\sqrt{6-2}}\times14.3\times10^{-3}= 0.0184\approx 0.02kg/s^2.\\[0.4cm]
        %u_{b_1}&=\frac{t_0.95}{\sqrt{n}}\sigma_{b_1}=\frac{2.57}{\sqrt{6}}\times1.19\times10^{-3}\approx12.5\times10^{-4}N.
    \end{split}
    \]
    
    Similarly, we obtain $u_{k_2}$ and $u_{k_3}$.
    \[
    \begin{split}
        \sigma_{k_2}&=0.008kg/s^2,\quad
        u_{k_2}=\frac{t_{0.95}}{\sqrt{n-2}}\sigma_{k_2}=0.01kg/s^2.\\
        %\sigma_{b_2}&=0.006N,\quad\quad\quad
        %u_{b_2}=\frac{t_{0.95}}{\sqrt{n}}\sigma_{b_2}=0.007N.\\
        \sigma_{k_3}&=0.013kg/s^2,\quad
        u_{k_3}=\frac{t_{0.95}}{\sqrt{n-2}}\sigma_{k_3}=0.017kg/s^2.\\
        %\sigma_{b_3}&=0.002N,\quad\quad\quad
        %u_{b_3}=\frac{t_{0.95}}{\sqrt{n}}\sigma_{b_3}=0.002N.\\
    \end{split}
    \]
    
    The relative uncertainty can be calculated by $u_r=u/\bar{X}\times100\%$.

    \textbf{For example},
    \[
        u_{r,k_1}=\frac{0.02}{2.22}\times100\%=0.9\%.
    \]
    Finally, the experimentally found $k_1$, $k_2$ and $k_3$ is
    \[
    \begin{split}
        k_1&= 2.22\pm 0.02 kg/s^2,\quad u_{r,k_1}=0.9\%\\
        k_2&= 2.25\pm 0.01 kg/s^2,\quad u_{r,k_2}=0.4\%\\
        k_3&= 1.11\pm 0.017kg/s^2,\quad u_{r,k_3}=1.5\%
    \end{split}
    \]

\subsubsection{Uncertainty of spring series' constants}
    To determine the theoretical value of $k_3$, we have the equations
    \[
    \begin{split}
        F&=k_1\Delta L_1,\\
        F&=k_2\Delta L_2,\\
        F&=k_3^{'}(\Delta L_1+\Delta L_2),
    \end{split}
    \]
    whose solution is $k_3^{'}=\frac{k_1k_2}{k_1+k_2}$. The theoretical value of $k_3$ is
    \[
        k_3^{'}=\frac{k_1k_2}{k_1+k_2}=\frac{2.22\times2.25}{2.22+2.25}=1.12kg/s^2.
    \]
    The propagated uncertainty of $k_3$ is estimated by the formula
    \[
    \begin{split}
        u_{k_3^{'}}&=\sqrt{(\frac{\partial k_3^{'}}{\partial k_1})^2(u_{k_1})^2+(\frac{\partial k_3^{'}}{\partial k_2})^2(u_{k_2})^2}\\[0.3cm]
        &=\sqrt{((\frac{k_2}{k_1+k_2})^2)^2(u_{k_1})^2+((\frac{k_1}{k_1+k_2})^2)^2(u_{k_2})^2}\\[0.3cm]
        &=\sqrt{(\frac{2.25}{2.22+2.25})^4(0.02)^2+(\frac{2.22}{2.22+2.25})^4(0.01)^2}\\[0.3cm]
        &=0.006kg/s^2
    \end{split}
    \]
    and the relative uncertainty is
    \[
        u_{rk_3^{'}}=\frac{u_{k_3^{'}}}{\overline{k_3^{'}}}=\frac{0.006}{1.12}=0.5\%.
    \]

    Hence the theoretical value of $k_3^{'}$ is
    \[
        k_3^{'}=1.12+0.006 kg/s^2,\quad u_{rk_3^{'}}=0.5\%.
    \]
    
    Compared with $k_3$ we obtain from curve fitting, we can calculate the deviation $\Delta k_3$ between $k_3$ and $k_3^{'}$ and the relative deviation $\Delta_r k_3$.
    \[ 
    \begin{split}
        \Delta k_3&=k_3-k_3^{'}=1.11-1.12=-0.01kg/s^2,\\
        \Delta_r k_3&=\frac{k_3-k_3^{'}}{k_3^{'}}=\frac{1.11-1.12}{1.12}=-0.9\%.
    \end{split}
    \]

    Recall that the 0.95-confidence interval of $k_3$ is $(1.093,1.127)$. The theoretical value $k_3^{'}=1.12+0.006 kg/s^2$ is indeed in this interval, which means the uncertainty analysis of $k_3$ is reasonable.\\
    
\subsection{Uncertainty of the slope of $T^2\ vs.\ M$}
    The uncertainty of one period is $u_T/10=1\times10^{-5}s$ and the uncertainty of mass is
    \[
    \begin{split}
        M&=m_{objI}+\frac{1}{3}m_{spr1}+\frac{1}{3}m_{spr2}+m_i,\\
        u_M&=\sqrt{
         (\frac{\partial M}{\partial m_{objI}})^2(u_{m_{objI}})^2
        +(\frac{\partial M}{\partial m_{spr1}})^2(u_{m_{spr1}})^2
        +(\frac{\partial M}{\partial m_{spr2}})^2(u_{m_{spr2}})^2
        +(\frac{\partial M}{\partial m_i})^2(u_{m_i})^2
        }\\
        &=\sqrt{
         (1)^2(u_{m_{objI}})^2
        +(\frac{1}{3})^2(u_{m_{spr1}})^2
        +(\frac{1}{3})^2(u_{m_{spr2}})^2
        +(1)^2(u_{m_i})^2
        }\\
        &=\sqrt{(u_{m_{objI}})^2+(u_{m_{spr1}})^2/9+(u_{m_{spr2}})^2/9+(u_{m_i})^2}.
    \end{split}
    \]

    \textbf{For example}, when $m_{objI}=(176.55\pm0.01)\times10^{-3}kg$, $m_{m_spr1}=(10.74\pm0.01)\times10^{-3}kg$, $m_{spr2}=(10.77\pm0.01)\times10^{-3}kg$ and $m_1=(4.83\pm0.01)\times10^{-3}kg$,
    \[
    \begin{split}
        u_M&=\sqrt{0.00001^2+0.00001^2/9+0.00001^2/9+0.00001^2}=0.0000149\\
        &\approx 1.5\times10^{-5}kg.
    \end{split}
    \]

    The uncertainty of $T^2$ is propagated uncertainty, which is calculated as
    \[
    \begin{split}
        &\frac{\partial T^2}{\partial T}=2T,\\
        &u_{T^2}=\sqrt{(2T)^2(u_T)^2}=2Tu_T.
    \end{split}
    \]

    \textbf{For example}, for $T=1.27989s$, $u_{T^2}=2\times1.27989\times1\times10^{-5}=2.55978\times10^{-5}\approx3\times10^{-5}s^2$. The complete results are listed in Table \ref{tmdata}, \ref{tmi1data} and \ref{tmi2data}.

    From the formula of period $T=2\pi\sqrt{\frac{M}{k}}$ we know that the slope of $T^2\ vs.\ M$ is
    \begin{equation}\label{slope}
        slope=\frac{T^2}{M}=\frac{4\pi^2}{k},
    \end{equation}
    where k is the effective spring constant. In this experiment, the effective spring constant can be calculated by
    \[ 
    \begin{split}
        F&=k_1\Delta x+k_2\Delta x,\\
        F&=k_{eff}\Delta x.
    \end{split}
    \]
    Thus we get $k_{eff}=k_1+k_2$. We take the reults of $k_1$ and $k_2$ as the theoretical value and ignore its uncertainty, then
    \begin{equation}\label{keff}
        k_{eff}=k_1+k_2=2.22+2.25=4.47kg/s^2.
    \end{equation}
    Plugging it into Eq. \ref{slope}, we get the theoretical slope $slope_t=4\times 3.14^2/4.47=8.82s^2/kg$
    By curve fitting, we get the slopes
    \[
    \begin{split}
        slope_{hor}&=8.72\pm 0.05s^2/kg,\quad u_{r,hor}=0.6\%,\\
        slope_{inc1}&=8.73\pm 0.14s^2/kg,\quad u_{r,inc1}=1.6\%,\\
        slope_{inc2}&=8.76\pm 0.17s^2/kg,\quad u_{r,inc2}=1.9\%,
    \end{split}
    \]
    where the uncertainty is calcualted by $t_0.95/\sqrt{6-2}\times \sigma$.
    Hence the experimental value of the slope is $slope_e=\overline{slope}=8.74s^/kg$.
    Similarly, we get the propagated and relative uncertainty of slope, which is
    \[
    \begin{split}
        u_{slope}&=\sqrt{(\frac{\partial slope}{\partial slope_{hor}})^2\cdot(u_{slope_{hor}})^2+(\frac{\partial slope}{\partial slope_{inc1}})^2\cdot(u_{slope_{inc1}})^2+(\frac{\partial slope}{\partial slope_{inc2}})^2\cdot(u_{slope_{inc2}})^2}\\
        &=\sqrt{(\frac{1}{3})^2\cdot(0.05)^2+(\frac{1}{3})^2\cdot(0.14)^2+(\frac{1}{3})^2\cdot(0.17)^2}=0.08s^2/kg.\\
        u_{r,slope}&=\frac{u_{slope}}{\overline{slope}}\times100\%=\frac{0.08}{8.74}\times100\%=0.9\%.
    \end{split}
    \]

    Hence, $slope_e=\overline{slope}=8.74\pm 0.08s^/kg, \quad u_{r,slope}=0.9\%$. Compared with the "theoretical" value of $slope_t=8.82s^2/kg$, we calculate the deviation $\Delta slope$ and relative deviation $\Delta_r slope$
    \[
    \begin{split}
        \Delta slope&=8.74-8.82=-0.08s^2/kg,\\
        \Delta_r slope&=\frac{8.74-8.82}{8.82}\times 100\%=-0.9\%.
    \end{split}
    \]
    The theoretical value is just on the boundary of uncertainty interval. It shows that our experimental values are relatively reliable.

\subsection{Uncertainty in the $v_{max}^2\ vs.\ A^2$ relation}
\subsubsection{Uncertainty of $\Delta x$}
    First, we need to determine the uncertainty of $x_{in}$ and $x_{out}$. The uncertainty of type-B of a calliper is $\Delta_{x,B}=\Delta{dev}=0.02\times10^{-3}m$. The distance is found by the average value of 3 measurements. To estimate  type-A uncertainty, the standard deviation of the average value is 
    \[
        s_{\overline{x_{in}}}=\sqrt{\frac{1}{n(n-1)}\sum_{i=1}^n(x_{in,i}-\overline{x_{in}})^2}.
    \]
    Using the data from Table \ref{x} we find that $s_{\overline{x_{in}}}\approx 0.0176\times10^{-3}m$. Considering $t_{0.95}=4.30$ for $n=3$, the type-A uncertainty is estimated as $\Delta_{x_{in},A}=4.30\times0.0176\times10^{-3}m\approx 0.0757\times10^{-3}m$.\\
    Hence the combined uncertainty is
    \[
         u_{x_{in}}=\sqrt{\Delta_{x_{in},A}^2+\Delta_{x_{in},B}^2}=\sqrt{(0.02\times10^{-3})^2+(0.0757\times10^{-3})^2}\approx 0.08\times10^{-3}m
    \]
     and the corresponding relative uncertainty is 
    \[
        u_{r,x_{in}}=\frac{u_{x_{in}}}{\overline{x_{in}}}\times 100\%=1.8\%.
    \]
    The experimentally found $x_{in}$ is 
    \[
        x_{in}=(4.49\pm 0.08) \times10^{-3}m,\quad u_{r,x_{in}}=1.8\%.
    \]
    Similarly, we know about $x_{out}$ that
    \[
    \begin{split}
        &\Delta_{x_{out},A}=t_{0.95}\cdot s_{\overline{x_{out}}}=0.0287\times10^{-3}m\\
        &\Delta_{x,B}=0.02\times10^{-3}m\\
        &u_{x_{out}}=0.03\times10^{-3}m,\quad
        u_{r,x_{out}}=0.2\%\\[0.4cm]
        &x_{out}=(15.41\pm 0.03) \times10^{-3}m,\quad u_{r,x_{out}}=0.2\%.
    \end{split}
    \]
    Then we can calculate the propagated uncertainty of $\Delta x$
    \[
    \begin{split}
        \frac{\partial\Delta x}{\partial x_{in}}&=\frac{\partial\Delta x}{\partial x_{out}}=\frac{1}{2},\\
        u_{\Delta x}&=\sqrt{(\frac{\partial\Delta x}{\partial x_{in}})^2(u_{x_{in}})^2+(\frac{\partial\Delta x}{\partial x_{out}})^2(u_{x_{out}})^2}\\
        &=\sqrt{(\frac{1}{2})^2(0.00008)^2+(\frac{!}{2})^2(0.00003)^2}\\
        &=0.04\times10^{-3}m,\\
        u_{r,\Delta x}&=\frac{u_{\Delta x}}{\Delta x}\times100\%=0.4\%\\[0.4cm]
        \Delta x&=(9.95\pm0.04)\times10^{-3}m,\quad u_{r,\Delta x}=0.4\%.
    \end{split}
    \]
\subsubsection{Uncertainty of the maximum speed $v_{max}$}
    Then we can calculate the propagated uncertainty of $v_{max}=\Delta x/\Delta t$. The partial derivatives are
    \[
    \begin{split}
        \frac{\partial v_{max}}{\partial \Delta x}&=\frac{1}{\Delta t}.\\[0.5cm]
        \frac{\partial v_{max}}{\partial \Delta t}&=-\frac{\Delta x}{(\Delta t)^2}.    
    \end{split}    
    \]
    Hence,
    \[
    \begin{split}
        u_{v_{max}}&=\sqrt{(\frac{\partial v_{max}}{\partial \Delta x})^2(u_{\Delta x})^2+(\frac{\partial v_{max}}{\partial \Delta t})^2(u_{\Delta t})^2}\\[0.4cm]
        &=\sqrt{(\frac{1}{\Delta t})^2(u_{\Delta x})^2+(-\frac{\Delta x}{(\Delta t)^2})^2(u_{\Delta t})^2}
    \end{split}
    \]

    In the case that $\Delta x=0.00995\pm 0.00004m$ and $\Delta t=0.04169\pm 0.00001s$,
    \[
    \begin{split}
        u_{v_{max}}&=\sqrt{(0.00004)^2/(0.04169)^2+(0.00995)^2(0.00001)^2/(0.04169)^4}\approx0.001m/s,\\
        u_{r,v_{max}}&=\frac{u_{v_{max}}}{v_{max}}\times100\%\approx0.4\%
    \end{split}
    \]
    \begin{table}[!h] \small
        \centering
        \begin{tabular}{|c|c|c|c|c|}
            \hline
            No. & $\Delta x[m]$ & $u_{\Delta x}[m]$ & $\Delta t[s]$ & $u_{\Delta t}[s]$\\ \hline
            1 & 0.00995 & 0.00004 & 0.04169 & 0.00001\\ \hline
            2 & 0.00995 & 0.00004 & 0.02061 & 0.00001\\ \hline
            3 & 0.00995 & 0.00004 & 0.01382 & 0.00001\\ \hline
            4 & 0.00995 & 0.00004 & 0.01045 & 0.00001\\ \hline
            5 & 0.00995 & 0.00004 & 0.00839 & 0.00001\\ \hline
            6 & 0.00995 & 0.00004 & 0.00703 & 0.00001\\ \hline
        \end{tabular}
        \caption{Data for the calculation of $v_{max}$}\label{dt}
    \end{table}

    Since the calculations are repeated and too complicated, the results of $v_{max}$ are listed in Table \ref{vadata1}.
    \begin{table}[!h] \small
        \centering
        \begin{tabular}{|c|c|c|c|}
            \hline
            No. & $v_{max}[m/s]$ & $u_{v_{max}}[m/s]$ & $u_{r,v_{max}}[\%]$\\ \hline
            1 & 0.239 & 0.001 & 0.4\\ \hline
            2 & 0.483 & 0.002 & 0.4\\ \hline
            3 & 0.720 & 0.003 & 0.4\\ \hline
            4 & 0.952 & 0.004 & 0.4\\ \hline
            5 & 1.19 & 0.005 & 0.4\\ \hline
            6 & 1.42 & 0.006 & 0.4\\ \hline
        \end{tabular}
        \caption{Results of $v_{max}$}\label{vadata1}
    \end{table}

    Then we can calculate the propagated uncertainty of $v_{max}^2$. The partial derivative is
    \[
        \frac{\partial v_{max}^2}{\partial v_{max}}=2v_{max}.
    \]
    Hence,
    \[
        u_{v_{max}^2}=\sqrt{(\frac{\partial v_{max}^2}{\partial v_{max}})^2(u_{v_{max}}})^2=2v_{max}u_{v_{max}}\\[0.4cm]
    \]

    \textbf{For example}, in the case that $v_{max}=0.239\pm 0.001m/s$,
    \[
    \begin{split}
        u_{v_{max}^2}&=2\times0.239\times0.001\approx0.0005m/s,\\
        u_{r,v_{max}^2}&=\frac{u_{v_{max}^2}}{v_{max}^2}\times100\%\approx0.8\%
    \end{split}
    \]
    The results of $v_{max}^2$ are shown in Table \ref{vadata2}.
    \begin{table}[!h] \small
        \centering
        \begin{tabular}{|c|c|c|c|}
            \hline
            No. & $v_{max}^2[m^2/s^2]$ & $u_{v_{max}^2}[m^2/s^2]$ & $u_{r,v_{max}^2}[\%]$\\ \hline
            1 & 0.0570 & 0.0005 & 0.8\\ \hline
            2 & 0.233 & 0.002 & 0.8\\ \hline
            3 & 0.518 & 0.004 & 0.8\\ \hline
            4 & 0.907 & 0.007 & 0.8\\ \hline
            5 & 1.41 & 0.012 & 0.8\\ \hline
            6 & 2.00 & 0.017 & 0.9\\ \hline
        \end{tabular}
        \caption{Results of $v_{max}$}\label{vadata2}
    \end{table}

    Now we will calculate the square of amplitudes $A^2$ The uncertainty of $A$ is $u_A=\Delta_{dev}=0.001m$.
    \[
    \begin{split}
        &\frac{\partial A^2}{\partial A}=2A.\\
        &u_{A^2}=\sqrt{(\frac{\partial A^2}{\partial A})^2(u_A)^2}=2Au_A\\[0.4cm]
    \end{split}
    \]

    \textbf{For example}, in the case that $A=0.050\pm 0.001m$,
    \[
    \begin{split}
        u_{A^2}&=2\times0.050\times0.001\approx0.0001m/s,\\
        u_{r,A^2}&=\frac{u_{A^2}}{A^2}\times100\%\approx4\%
    \end{split}
    \]
    The results of $A^2$ are shown in Table \ref{adata2}.
    \begin{table}[!h] \small
        \centering
        \begin{tabular}{|c|c|c|c|}
            \hline
            No. & $A^2[m^2]$ & $u_{A^2}[m^2]$ & $u_{r,A^2}[\%]$\\ \hline
            1 & 0.0025 & 0.0001 & 4\\ \hline
            2 & 0.0100 & 0.0002 & 2\\ \hline
            3 & 0.0225 & 0.0003 & 1.3\\ \hline
            4 & 0.0400 & 0.0004 & 1.0\\ \hline
            5 & 0.0625 & 0.0005 & 0.8\\ \hline
            6 & 0.0900 & 0.0006 & 0.6\\ \hline
        \end{tabular}
        \caption{Results of $A^2$}\label{adata2}
    \end{table}

    Finally, by curve fitting we find the relation is that $v_{max}^2\propto A^2$. The slope is
    \begin{equation}\label{k/m}
        slope=22.22\pm0.31s^{-2}, \quad u_r=1.4\%.
    \end{equation}

    According to the formula of energy conservation in simple harmonic motion $\frac{1}{2}mv_{max}^2=\frac{1}{2}kA^2$, we obtain the theoretical value of the slope $k/m$ from the experimentally found $k$ (from Eq. \ref{keff}) and $m$ (from Table \ref{M}), which will be regarded as theoretical value. Here $m$ is the mass of the object with U-shape shutter and springs.
    \[
    \begin{split}
        k&=k_{eff}=4.47kg/s^2\\
        m&=m_{objU}=0.19392kg\\
        \frac{k}{m}=\frac{4.47}{0.19392}=23.1s^{-2}.
    \end{split}
    \]
    Compared with the fitting slope in Eq. \ref{k/m}, we can calculate the deviation $\Delta slope$ and relative deviation $\Delta_r slope$
    \[
    \begin{split}
        \Delta slope&=22.2-23.1=-0.9s^2/kg,\\
        \Delta_r slope&=\frac{22.2-23.1}{23.1}\times 100\%=-4\%.
    \end{split}
    \]

    We see that the theoretical value is beyond the boundary of uncertainty interval. It shows that our experimental values are too small and not relatively reliable. The reason will be further discussed in section 6.
