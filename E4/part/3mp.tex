\section{Measurement Procedure}

\subsection{Resonance method}

\begin{enumerate}
    \item Set the initial distance between $S_1$ and $S_2$ at about $1cm$.
    \item Turn on the signal source and the oscilloscope. Then set the
      following options on the panel of the signal source. 
    \begin{enumerate}
        \item Choose \emph{Continuous} wave for \emph{Method}.
        \item Adjust Signal Strength until a $10V$ peak voltage is observed
          on the oscilloscope. 
        \item Adjust Signal Frequency between 34.5 kHz and 40 kHz until the
          peak-to-peak voltage reaches its maximum. Record the frequency. 
    \end{enumerate}
    \item Increase $L$ gradually by moving $S_2$, and observe the output
      voltage of $S_2$ on the oscilloscope. Record the position of $S_2$ as
      $L_2$ when the output voltage reaches an maximum. 
    \item Repeat step 3 to record 20 values of $L_2$ and calculate $v$.
\end{enumerate}

\subsection{Phase-comparison method}

\begin{enumerate}
    \item Use Lissajous figures to observe the phase difference between the
      transmitted and the received signals. Move $S_2$ and record the
      position when the Lissajous figure becomes a straight line with the
      same slope. 
    \item Repeat step 1 to collect 12 sets of data. Use the successive
      difference method to process the data and calculate $v$. 
\end{enumerate}

\subsection{Time-difference method (liquid)}
Since the pulse wave causes damped oscillations at the receiver, there will
be significant interference if $S_1$ and $S_2$ resonate. The resonance can
be observed on the oscilloscope. 

\begin{enumerate}
    \item Choose Pulse Wave for Method on the panel of the signal source.
    \item Adjust the frequency to $100$ Hz and the width to $500 \mu s$.
    \item Use the cursor function of the oscilloscope to measure the time
      and the distance between the the starting points of neighboring
      periods. 
    \item Record the distance $L_1$ and the time $t_1$.
    \item Move $S_2$ to another position and repeat step 3. Record $L_i$ and
      $t_i$, $i = 2, 3, 4, \cdots$. 
    \item Repeat step 4 to collect 12 pairs of $L_i$ and $t_i$. Plot the
      $L_i\ vs.\ t_i$ graph and use computer software to find a linear fit
      to the data. The slope of the line is the speed $v_{water}$. 
\end{enumerate}
