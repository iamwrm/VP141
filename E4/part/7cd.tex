\section{Conclusion and Discussion}

For the resonance method and the comparison method
    
In the experiment 
the speed of sound in the air was measured through two ways: 
the resonance method and the phase comparison method. 
From the two means, we have 
    \begin{equation}
    \begin{split}
        v&=351.05\pm0.4 m/s,\quad u_{r,v}=0.10\%;\\
        v&=348.95\pm10 m/s,\quad u_{r,v}=3\%,
    \end{split}
    \end{equation}
    respectively. 

    From the result after analyzing, we can conclude that the results of the resonance method has higher precision but not enough accuracy.
    On the other hand, the results of the phase comparison method has higher accuracy but not enough precision.

The reason account for the lack of  preciseness may be the easiness to 
determine the time we get the distance which shares the same length of half wave length for Resonance Method, 
or one wave length for Phase comparison Method.

For Resonance Method, it's not very easy to observe whether the wave on oscilloscope became the maximum amplitude. 
For Phase-comparison Method, it is easy to observe whether the Lissajous figure becomes a straight line with a positive slope in each period. 

Thus we managed to get the data that are close to each other because of the easy observation. 
The very uncertainty may due to random error in this experiment.

The reason for the preciseness may be that we manage to obverse the place of first resonance point, though the resonance curve is not flat enough, which may make the observation difficult.
 The little relative error may come from random error in this experiment and the difficulty to obverse.

In conclusion, the result produced in the experiment is relatively precise and accurate. 
Through the Resonance Method and the Phase-comparison Method, we can get a very precise result. 
Through the Time-difference Method, we can get a very accurate result, though it may only be an accident in this experiment. In order to compare these four methods, we should do the experiment for several times to reduce random error.
Moreover, we are introduced the Successive Difference Method, which is a more effective way to reduce error. We can use this method in other experiments.
